\documentclass{article}
\usepackage[utf8]{inputenc}
\usepackage{amsmath}
\usepackage{scrextend}
\usepackage{setspace}
\usepackage{amsfonts}
\usepackage{amssymb}
\usepackage{mathrsfs}
\usepackage{graphicx}
\graphicspath{ {../../images/} }

\title{Fundamental Concepts of Analysis\\
\large{Week 1 Notes (d)}}
\author{shaozewxy }
\date{September 2022}

\doublespacing
\begin{document}

\maketitle
\section*{Summary}
A list of basic theorem for sequences.
\section*{10 Sequences of Real Numbers}
\subsection*{Definition of sequences}
Given $X$ a set. A \textbf{sequence} of elements of $X$ is a function from $\mathbf{P}$ the set of positive integers to $X$.\\
So \textbf{a real sequence} is a function from $\mathbf{P} \rightarrow \mathbf{R}$.\\
The notion is $\{a_n\}_{n=1}^{\infty}$ where $a$ denotes the function and $a_n$ is the value of the function at $n \in \mathbf{P}$.\\
Can also use $a_1, a_2, ...$ or $\{a_n\}$ to denote a sequence.
\subsection*{Definition of limit of a sequence}
Given $\{a_n\}$ a sequence, we say that $\{a_n\}$ has \textbf{limit} $L \in \mathbf{R}$ if $\forall \epsilon > 0, \exists N \in \mathbf{P}$ such that $n \geq N \rightarrow |a_n - L| < \epsilon$.
\subsection*{10.3 Limit is unique}
\textbf{Proof:}
\begin{addmargin}[1em]{0em}
    Suppose $L \neq L'$ both limits of $\{a_n\}$.\\
    Then choose $\epsilon = |L - L'|/4$, we have that $\exists N_1$ such that $\forall n > N_1 \in \mathbf{P}, |L - a_n| < \epsilon$, similarly $N_2$ exists for $L'$, we choose $N$ to be the larger of $N_1, N_2$.\\
    Then choose $n > N$, we have $|L - a_n| < \epsilon, |L' - a_n| < \epsilon$.\\
    However, it is obvious such $a_n$ doesn't exist:\\
    \includegraphics*[]{UniqueLimit.png}
\end{addmargin} 
Now we can denote this unique limit of $\{a_n\}$ as $\lim_{n \rightarrow \infty} a_n = L$.
\subsection*{Test of non-limit}
From the definition we know that $\{a_n\}$ doesn't have a limit $L$ if $\exists \epsilon$ such that there are infinitely many $n \in \mathbf{P}$ such that $|L - a_n| > \epsilon$.

\section*{11 Subsequences}
\subsection*{Definition of subsequence}
Given $\{a_n\}$ a sequence. Define a $f: \mathbf{P} \rightarrow \mathbf{P}$ to be strictly increasing. The sequence $\{a_{f(n)}\}_{n=1}^\infty$ is called a \textbf{subsequence} of the sequence $\{a_n\}$.
\subsection*{11.2 Subsequences share the limit}
Given $\{a_n\}$ with limit $L$, then any subsequence of $\{a_n\}$ also has limit $L$.\\
\textbf{Proof:}
\begin{addmargin}[1em]{0em}
    $\forall \epsilon > 0$ we know that $\exists N$ such that $\forall n > N \in \mathbf{P}, |a_n - L| < \epsilon$.\\
    Then given an strictly increasing $f: \mathbf{P} \rightarrow \mathbf{P}$, we know $\exists N'$ such that $f(N') > N$.\\
    Then we say that with this $N', \forall n \in \mathbf{P} > N', a_{f(n)} = a_{n'}$ for some $n' = f(n) > N' > N, \rightarrow |L - a_{f(n)}| < \epsilon$.
\end{addmargin}

\section*{12 Algebra of Limits}
We say that $\{a_n\}$ is \textbf{convergent} if it has a limit and \textbf{divergent} if it doesn't have a limit.
\subsection*{12.1 Unchanged sequence converges}
The proof of this is obvious.
\subsection*{12.2-12.9  Operation of limits}
These operations of convergent sequences reflect to their limits:\\
Addition, subtraction, multiplication with number, multiplication, division (given it makes sense).\\
Here the case of division needs to be taken special care of.\\
Suppose $\lim_{n \rightarrow \infty}a_n = L \neq 0$, then there can \textbf{only be finitely many} $a_n = 0$.\\
This comes from the fact that $L$ is the limit of $\{a_n\}$, so as $n$ increases, $\{a_n\}$ will contract into a smaller range centering on $L$, only finitely many $n$ outside of this range and could be $=0$.\\
Then we can define $\frac{1}{a_n}$ as a sequence with limit $\frac{1}{L}$.
\section*{13 Bounded Sequences}
\subsection*{Definition of bounded}
A sequence $\{a_n\}$ is \textbf{bounded above(below)} if $\exists M$ such that $\forall n \in \mathbf{P} a_n \leq M(a_n \geq M)$.\\
From this definition we can see that $\{a_n\}$ is bounded $\iff \exists M \mathbf{P}$ such that $\forall n \in \mathbf{P}, |a_n| \leq M$.\\
Therefore we have
\subsection*{13.2 Convergent sequences are bounded}
\subsection*{13.3 Multiplication of bounded sequeces}
Given $\{a_n\}, \{b_n\}$ with $\lim_{n \rightarrow \infty}a_n = 0$ and $\{b_n\}$ bounded, then we have $\lim_{n \rightarrow \infty}a_nb_n = 0$.
\end{document}