\documentclass{article}
\usepackage[utf8]{inputenc}
\usepackage{amsmath}
\usepackage{scrextend}
\usepackage{setspace}
\usepackage{amsfonts}
\usepackage{amssymb}
\usepackage{mathrsfs}

\title{Fundamental Concepts of Analysis\\
\large{Week 1 Notes (b)}}
\author{shaozewxy }
\date{June 2022}

\doublespacing
\begin{document}

\maketitle
\section*{5 Least-Upper-Bound Axiom}
\subsection*{5.1 Definition of lower/upper bound}
A nonempty subset $X \subseteq \mathbf{R}$ is said to be \textbf{bounded above(below)} if there exists a real number $a$ such that $\forall x \in X, x \leq a(x \geq a)$. The number $a$ is called an \textbf{upper(lower) bound} for $X$.
\subsection*{5.2 Definition of least-upper(greatest-lower) bound}
Let $X \subseteq \mathbf{R}$ be nonempty.\\
A number $a \in \mathbf{R}$ is said to be a \textbf{least upper bound} for $X$ if
\begin{itemize}
    \item[(i)] $a$ is an upper bound for $X$.
    \item[(ii)] If $b$ is an upper bound for $X$, then $a \leq b$.
\end{itemize}
The \textbf{greatest lower bound} is defined similarly.\\
Also, the (ii) part can be stated as:
\begin{addmargin}[1em]{0em}
    If $b < a$, then $b$ is not an upper bound for $X$.
\end{addmargin}
Using 5.1, the statement above is equivalent to
\begin{itemize}
    \item[(ii')] If $b < a$, then $\exists x \in X, x > b$
\end{itemize}
Sometimes when proving least upper bound, using (ii') is more convenient.
\subsection*{5.3 Least upper bound is unique}
Let $X \subseteq \mathbf{R}$. If $a, b$ are least upper bounds of $X$, then $a = b$.\\
\textbf{Proof:}
\begin{addmargin}[1em]{0em}
    Suppose $a < b$, then by definition $\exists a < b$ such that $a$ is an upper bound, therefore $b$ is not an upper bound. Contradiction.
\end{addmargin}
We can then denote the least-upper(greatest-lower) bound of a set $X$ by
\begin{equation*}
    lub\ X = sup\ X, glb\ X = inf\ X
\end{equation*}
\subsection*{Axiom 13}
A nonempty subset of real numbers bounded above has a least upper bound.\\
This axiom is essentially saying there are no holes in the real line.
\subsection*{5.4 Conditions for GLB}
A nonempty subset of real numbers bounded below has a greatest lower bound.\\
\textbf{Proof:}
\begin{addmargin}[1em]{0em}
    Denote the set $X \subseteq \mathbf{R}$ which is bounded below. Then we create $Y$ as the set of real numbers that are lower bounds of $X$.\\
    Since $X$ non-empty, $\exists x \in X$ such that $x$ is an upper bound of $Y$.\\
    Therefore $Y$ bounded above and $\exists y = lub\ Y$.\\
    We claim that $a = lub\ Y = glb\ X$:\\
    First NTS $a$ is a lower bound of $X$:\\
    Suppose $\exists x \in X$ such that $x < a$, then since $a = lub\ Y$ we know that $\exists y \in Y$ such that $y > x$ but since $y$ is a lower bound of $X$ this is a contradiction. Therefore no such $x$ exists, and thus $a$ is a lower bound of $X$.\\
    Then NTS $a$ is the greaster upper bound of $X$:\\
    $\forall b > a$ clearly $b \notin Y$ and therefore $b$ not a lower bound. Therefore $a$ is the greatest lower bound.
\end{addmargin}
\section*{6 Set of Positive Integers}
\subsection*{6.1 Definition of sucessor set}
A subset $X \subseteq \mathbf{R}$ is said to be a \textbf{successor set}
\begin{itemize}
    \item[(i)] If $1 \in X$.
    \item[(ii)] If $n \in X$, then $n + 1 \in X$.
\end{itemize}
Since $\mathbf{R}$ itself is a successor set, we know that successor set exists.
\subsection*{6.2 Intersection of successor sets}
If $\mathscr{A}$ is a nonempty collection of successor sets, then $\cap \mathscr{A}$ is a successor set.\\
\textbf{Proof:}
\begin{addmargin}[1em]{0em}
    For (i), $\forall A \in \mathscr{A}$, since $1 \in A \rightarrow 1 \in \cap \mathscr{A}$.\\
    For (ii), $\forall n \in \cap\mathscr{A}$, since $\forall A \in \mathscr{A}, n \in A \rightarrow n+1 \in A$, we know that $n+1 \in \cap \mathscr{A}$.
\end{addmargin}
\subsection*{6.3 Definiton of positive integers}
The set $\mathbf{P}$ of \textbf{positive integers} is the intersection of the family of all successor sets.\\
The meaning of the above defintion is that $\mathbf{P}$ is the smallest successor set:\\
If $X$ is a successor set, then $\mathbf{P} \subset \mathbf{X}$.
\subsection*{6.4 Mathematical Induction}
Suppose that for each positive integer $n$ we have a statement $S(n)$ and that
\begin{itemize}
    \item[(i)] $S(1)$ is true.
    \item[(ii)] If $S(n)$ is true, then $S(n+1)$ is true.
\end{itemize}
Then $S(n)$ is true for every positive integer $n$.\\
\textbf{Proof:}
\begin{addmargin}[1em]{0em}
    Let $G = \{n \in \mathbf{P} | S(n) \textrm{ is true}\}$, then $G \subseteq \mathbf{P}$.\\
    But by definition $1 \in G$ and $n \in G \rightarrow n+1 \in G$, therefore $G$ a successor set and $\mathbf{P} \subseteq G$.\\
    Therefore $G = \mathbf{P}$.
\end{addmargin}
Here are two examples of the induction:
\subsection*{6.5 Positive integer $\geq 1$}
If $n$ is a positive integer, then $n \geq 1$.\\
\textbf{Proof:}
\begin{addmargin}[1em]{0em}
    For (i), clearly $1 \geq 1$.\\
    For (ii), suppose $n \geq 1$, then $n+1 \geq n \geq 1$.\\
    Therefore this is true for all positive integer $n$.
\end{addmargin}
\subsection*{6.6 Sum of positive integers are positive integers}
If $m, n \in \mathbf{P}$ then $m + n \in \mathbf{P}$.\\
\textbf{Proof:}
\begin{addmargin}[1em]{0em}
    Define $S(m)$ to be "$\forall n \in \mathbf{P}, m+n \in \mathbf{P}$".\\
    For (i), it is by definiton that $\forall n \in \mathbf{P}, 1+n \in \mathbf{P}$.\\
    For (ii), $\forall n \in \mathbf{P} m+1+n = (m+n) + 1$, since $m+n \in \mathbf{P} \rightarrow (m+n) + 1 \in \mathbf{P}$.\\
    Therefore this is true for all positive integer $m$.
\end{addmargin}
We then prove the well-ordering Theorem
\subsection*{6.7}
If $n \in \mathbf{P}$, then either $n - 1 = 0$ or $n - 1 \in \mathbf{P}$.\\
The proof of this is obvious.
\subsection*{6.8}
If $m, n \in \mathbf{P}$ and $m < n$, then $n - m \in \mathbf{P}$.\\
\textbf{Proof:}
\begin{addmargin}[1em]{0em}
    Let $S(m)$ be the statement above.\\
    For (i), if $1 < n$ then $n - 1$ by 6.7 either $n - 1 = 0$ or $n - 1 \in \mathbf{P}$. Since $n \neq 1$, $n - 1 \in \mathbf{P}$.\\
    For (ii), if $m+1 < n$, then $n - (m+1) = n - m - 1$ by 6.7 and the fact that $n - m \in \mathbf{P}$, either $n - m - 1 = 0$ or $n - m - 1 \in \mathbf{P}$. Since $n \neq m+1$, we know that $n - m - 1 \in \mathbf{P}$.\\
    THereofre this is true for all positive integers.
\end{addmargin}
\subsection*{6.9}
Let $n$ be a positive integer. No positive integer $m$ satisfies $n < m < n+1$\\
\textbf{Proof:}
\begin{addmargin}[1em]{0em}
    By 6.8 we know that $m - n \in \mathbf{P}$, but since $m < n + 1 \rightarrow m - n < 1$. This contradicts with the fact that $m - n \in \mathbf{P}$ should be $ \geq 1$.\\
    Therefore no such $m$ exists.
\end{addmargin}
\subsection*{6.10 Well-Ordering Theorem}
If $X$ is a non-void subset of positive integers, then $X$ contains a least element $a \in X$ such that $\forall x \in X, a \leq x$.\\
\textbf{Proof:}
\begin{addmargin}[1em]{0em}
     We let the statement $S(n)$ be that "If $n \in X$, then $X$ contains a least element".\\
     For (i), if $1 \in X$ then $1$ is the least element by 6.5.\\
     For (ii), if $S(n)$ is true, i.e. $n \in X \rightarrow X$ contains a least element, then suppose $n+1 \in X$, WTS that $X$ has a least element.\\
     Now $X \cup \{n\}$ has a least element $m$. If $m \in X$ then we are done, therefore assume $m \notin X \rightarrow m = n$.\\
     This means that $\forall x \in X, n \leq x$. Therefore by 6.9 we have $\forall x \in X, n+1 \leq x$, therefore $n+1$ is the least element of $X$.
\end{addmargin}
\subsection*{6.11}
The set of positive integers is not bounded above.\\
\textbf{Proof:}
\begin{addmargin}[1em]{0em}
    If $\mathbf{P}$ is bouned above, then $lub\ \mathbf{P} = a$.\\
    We know that $a - 1 < a$ is not an upper bound of $\mathbf{P}$ and therefore $\exists n \in \mathbf{P}$ such that $a - 1 < n \rightarrow a < n+1 \in \mathbf{R}$. Contradiction.
\end{addmargin}
The ordering of positive integers also implies some ordering of positive real numbers:
\subsection*{6.12 Archimedean ordering of positive real numbers}
If $a, b$ positive real numbers, then $\exists n \in \mathbf{P}$ such that $a < nb$.\\
\textbf{Proof:}
\begin{addmargin}[1em]{0em}
    Since $\mathbf{P}$ is not bounded above, then $\exists n \in \mathbf{P}$ such that $a/b < n$.\\
    Therefore $a < nb$.
\end{addmargin}
\end{document}