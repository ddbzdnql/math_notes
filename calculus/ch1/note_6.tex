\documentclass{article}
\usepackage[utf8]{inputenc}
\usepackage{amsmath}
\usepackage{scrextend}
\usepackage{setspace}
\usepackage{amsfonts}
\usepackage{amssymb}
\usepackage{graphicx}
\graphicspath{ {../../images/} }

\title{Calculus\\
\large{Notes on 2023/06/15}}
\author{shaozewxy }
\date{June 2023}

\doublespacing
\begin{document}

\maketitle
\section*{1.8 Rules for derivatives}
This chapter mainly uses the theorems to derive some results useful for calculating derivatives.
\subsection*{1.8.1 Basic rules}
Suppose $U \subseteq \mathbb{R}^n$ is open, then
\begin{enumerate}
    \item If $f: U \rightarrow \mathbb{R}^m$ is constant, then $f$ is differenciable and $\mathrm{D}f = [0]$.
    \item If $f: \mathbb{R}^n \rightarrow \mathbb{R}^m$ is linear, then $f$ is differenciable everywhere, and $[\mathrm{D}f(\mathbf{a})]\mathbf{v} = f(\mathbf{v})$.
    \item Given $f_1, ..., f_m: U \rightarrow \mathbb{R}$ all differenciable at $a$, then
    \begin{equation*}
        f = \begin{pmatrix}
            f_1\\
            ...\\
            f_m
        \end{pmatrix}: U \rightarrow \mathbb{R}^m
    \end{equation*}
    is also differentiable at $a$ and
    \begin{equation*}
        [\mathrm{D}f(a)]v = \begin{bmatrix}
            [\mathrm{D}f_1(a)]v\\
            ...\\
            [\mathrm{D}f_m(a)]v\\
        \end{bmatrix}
    \end{equation*}
    \item Given $f, g: U \rightarrow \mathbb{R}^m$ differenciable at $a$, then so is $f+g$, and $[\mathrm{D}(f+g)(a)] = [\mathrm{D}f(a)]+[\mathrm{D}g(a)]$.
    \item Given $f: U \rightarrow \mathbb{R}, g: U \rightarrow \mathbb{R}^m$ differenciable at $a$, then so is $fg$, and $[\mathrm{D}(fg)(a)]v = ([\mathbf{D}f(a)]v)g(a) + f(a)([\mathrm{D}g(a)]v)$.
    \item Given $f: U \rightarrow \mathbb{R}, g: U \rightarrow \mathbb{R}^m$ both differenciable at $a$ and $f(a) \ne 0$, then
    \begin{equation*}
        [\mathrm{D}(\frac{g}{f})(a)]v = \frac{[\mathrm{D}g(a)]v}{f(a)} - \frac{([\mathrm{D}f(a)]v)g(a)}{(f(a))^2}
    \end{equation*} 
    \item Given $f, g: U \rightarrow \mathbb{R}^m$ both differenciable at $a$, then so is $f \cdot g: U \rightarrow \mathbb{R}$ with $[\mathrm{D}(f\cdot g)(a)]v = ([\mathrm{D}f(a)]v)\cdot g(a) + f(a)\cdot ([\mathrm{D}g(a)]v)$
\end{enumerate}
\textbf{Proof:}
\begin{addmargin}[10px]{0px}
    We employ the equality
    \begin{equation*}
        \lim_{\overrightarrow{h}\rightarrow \overrightarrow{0}} \frac{1}{|h|}(f(a+h)-f(a) - [\mathrm{D}f(a)]h) = 0
    \end{equation*}
    Using 2. as an example:\\
    Given $f$ is linear, we know that $f(a+h) = f(a) + f(h)$, therefore,
    \begin{equation*}
        \begin{split}
            &\lim_{\overrightarrow{h}\rightarrow \overrightarrow{0}} \frac{1}{|h|}(f(a+h) - f(a) - f(h))\\
            =& \frac{1}{|h|}(f(a) + f(h) - f(a) - f(h)) = 0
        \end{split}
    \end{equation*}
    Using this equality, 1 to 4 are easily proven.\\
    For 5,
    \begin{equation*}
        \begin{split}
            \lim_{h \rightarrow 0}& \frac{1}{|h|}((fg)(a+h) - (fg)(a) - [\mathrm{D}(fg)(a)]h)\\
            =& \frac{1}{|h|}(f(a+h)g(a+h) - f(a)g(a) - f(a)g(a+h) + f(a)g(a+h) - [\mathrm{D}(fg)(a)]h)\\
            =& \frac{1}{|h|}((f(a+h)-f(a))g(a+h) + f(a)(g(a+h)-g(a)) - f(a)([\mathrm{D}g(a)]h) - g(a)([\mathrm{D}f(a)]h))\\
            =& \frac{1}{|h|}((g(a+h)-g(a)-[\mathrm{D}g(a)]h)f(a) - (f(a+h) - f(a) - [\mathrm{D}f(a)]h)g(a)\\
            &+ ([\mathrm{D}f(a)]h)(g(a+h) - g(a)))
        \end{split}
    \end{equation*}
    By definition, the first two terms vanish as $h \rightarrow 0$. We only NTS that
    \begin{equation*}
        \lim_{h \rightarrow 0} \frac{[\mathrm{D}f(a)h]}{|h|}(g(a+h) - g(a)) = 0
    \end{equation*}
    We use the fact that the first term is bounded while the second vanishes and therefore the whole term vanishes.\\
    The fact that the second term vanishes comes from the fact that $g$ is differenciable and therefore continuous at $a$.\\
    For the first term:
    \begin{equation*}
        \begin{split}
            \left|\frac{[\mathrm{D}f(a)]h}{|h|}\right| &\leq |\mathrm{D}f(a)| \left|\frac{h}{|h|}\right| = |\mathrm{D}f(a)|
        \end{split}
    \end{equation*}
    Therefore the first term is bounded and the whole term vanishes.\\
    For 6,
    6 comes from 5 and we only NTS that
    \begin{equation*}
        \left[\mathrm{D}\frac{1}{f}(a)\right]v = - \frac{[\mathrm{D}f(a)]v}{(f(a))^2}
    \end{equation*}
    Here we have
    \begin{equation*}
        \begin{split}
            \lim_{h \rightarrow 0}& \frac{1}{|h|}\left(
                \frac{1}{f(a+h)} - \frac{1}{f(a)} + \frac{[\mathrm{D}f(a)]h}{(f(a))^2}
            \right)\\
            =&\frac{1}{|h|}\left(
                \frac{[\mathrm{D}f(a)]h - f(a+h) + f(a)}{(f(a))^2} + \frac{f(a+h)-f(a)}{(f(a))^2} - \frac{f(a+h) - f(a)}{f(a+h)f(a)}
            \right)
        \end{split}
    \end{equation*}
    The first term by definition vanishes, the second and third term can be rewritten as:
    \begin{equation*}
        \frac{f(a+h)-f(a)}{|h|}\left(\frac{1}{f(a)}\left(\frac{1}{f(a)} - \frac{1}{f(a+h)}\right)\right)
    \end{equation*}
    Since we already know $\frac{1}{|h|}[\mathrm{D}f(a)]h$ is bounded and that $\frac{1}{|h|}\left(
        [\mathrm{D}f(a)]h - (f(a+h) - f(a))
    \right)$ vanishes, therefore the first term also vanishes.
    
\end{addmargin}
\end{document}