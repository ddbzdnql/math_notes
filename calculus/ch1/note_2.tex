\documentclass{article}
\usepackage[utf8]{inputenc}
\usepackage{amsmath}
\usepackage{scrextend}
\usepackage{setspace}
\usepackage{amsfonts}
\usepackage{amssymb}
\usepackage{graphicx}
\graphicspath{ {../../images/} }

\title{Calculus\\
\large{Notes on 2023/06/09}}
\author{shaozewxy }
\date{June 2023}

\doublespacing
\begin{document}

\maketitle

\section{Geometry of $\mathbb{R}^n$ (Ch 1.4)}
\textbf{Algebra is all about eqaulity, calculus is all about inequality.}
\subsection{Determinants and Traces}
Currently we only define determinants and traces in $\mathbb{R}^2$.
\begin{equation*}
    A = \begin{bmatrix}
        a_1& b_1\\
        a_2& b_2
    \end{bmatrix}, \det A = a_1b_2 - b_1a_2, \mathrm{tr} A = a_1 + b_2
\end{equation*}
\textbf{Geometric interpretation of determinants}
\begin{equation*}
    \forall \mathbf{a, b} \in \mathbb{R}^2, \det [\mathbf{a, b}] = \textrm{Area of parallelogram by} \mathbf{a, b}
\end{equation*}
\textbf{Proof:}
\begin{addmargin}[10px]{0px}
    We know that $\sin \theta = \sqrt{1 - \cos^2 \theta}$, and $\cos \theta = \frac{\mathbf{ab}}{\mathbf{|a||b|}}$, therefore,
    \begin{equation*}
        \begin{split}
            \sin \theta &= \sqrt{1 - \left(\frac{\mathbf{ab}}{\mathbf{|a||b|}}\right)^2}\\
            &= \sqrt{1 - \frac{(a_1b_1 + a_2b_2)^2}{(a_1^2+a_2^2)(b_1^2+b_2^2)}}\\
            &= \sqrt{\frac{a_1^2b_1^2+a_1^2b_2^2+a_2^2b_1^2+a_2^2b_2^2 - a_1^2b_1^2 - a_2^2b_2^2 - 2a_1a_2b_1b_2}
            {a_1^2b_1^2+a_1^2b_2^2+a_2^2b_1^2+a_2^2b_2^2}}\\
            &= \sqrt{\frac{(a_2b_1-a_1b_2)^2}{(\mathbf{|a||b|})^2}}\\
            &= \frac{|\det [\mathbf{a, b}]|}{\mathbf{|a||b|}}
        \end{split}
    \end{equation*}
    i.e., $|\det[\mathbf{a, b}]| = \sin \theta \cdot \mathbf{|a||b|}$
\end{addmargin}
Given a standard $\mathbb{R}^2$, $\det [\mathbf{a, b}]$ is positive $\iff \mathbf{a}$ counterclk-wise from $\mathbf{b}$.\\
\textbf{Proof:}
\begin{addmargin}[10px]{0px}
    Define $\mathbf{c}$ by rotating $\mathbf{a}$ by $\frac{\pi}{2}$, i.e. $\mathbf{c} = \begin{bmatrix}
        -a_2\\
        a_1
    \end{bmatrix}$, then we have $\mathbf{bc} = a_1b_2 - a_2b_1 = \det [\mathbf{a, b}]$.\\
    \begin{equation*}
        \det [\mathbf{a, b}] > 0 \Rightarrow \theta < \frac{\pi}{2} \Rightarrow \theta + \frac{\pi}{2} < \pi
    \end{equation*}
    i.e., $\mathbf{a}$ counterclk-wise from $\mathbf{b}$.
\end{addmargin}
\subsection*{Determinants and cross products in $\mathbb{R}^3$}
\textbf{Determinants in $\mathbb{R}^3$}
\begin{equation*}
    \det \begin{bmatrix}
        a_1& b_1& c_1\\
        a_2& b_2& c_2\\
        a_3& b_3& c_3
    \end{bmatrix} = a_1 \det \begin{bmatrix}
        b_2& c_2\\
        b_3& c_3
    \end{bmatrix} - a_2 \det \begin{bmatrix}
        b_1& c_1\\
        b_3& c_3
    \end{bmatrix} + a_3 \det \begin{bmatrix}
        b_2& c_2\\
        b_3& c_3
    \end{bmatrix}
\end{equation*}
\textbf{Cross product in $\mathbb{R}^3$}
\begin{equation*}
    \forall \mathbf{a, b} \in \mathbf{R}^3, \mathbf{a \times b} = \textrm{components of } \det \begin{bmatrix}
        1& a_1& b_1\\
        1& a_2& b_2\\
        1& a_3& b_3
    \end{bmatrix} = \begin{bmatrix}
        \det \begin{bmatrix}
            a_2& b_2\\
            a_3& b_3
        \end{bmatrix}\\
        -\det \begin{bmatrix}
            a_1& b_1\\
            a_3& b_3
        \end{bmatrix}\\
        \det \begin{bmatrix}
            a_1& b_1\\
            a_2& b_2
        \end{bmatrix}
    \end{bmatrix}
\end{equation*}
From the definition we can easily see that
\begin{equation*}
    \det [\mathbf{a, b, c}] = \mathbf{a} \cdot (\mathbf{b\times c})
\end{equation*}
\textbf{Geometric interpretation of cross product}
\begin{enumerate}
    \item $\mathbf{a \times b}$ orthogonal to the plane spanned by $\mathbf{a, b}$.
    \item $|\mathbf{a\times b}|$ is the area of the parallelogram between $\mathbf{a, b}$.
    \item $\mathbf{a, b}$ no colinear $\iff \det[\mathbf{a, b, a\times b}] > 0$
\end{enumerate}
\textbf{Proof:}
\begin{addmargin}[10px]{0px}
    The first one is easy to verify by multiplying $\mathbf{a\cdot (a\times b)}$ and so on.\\
    For 2, we use $\sin \theta = \sqrt{1 - \cos^2 \theta}$:
    \begin{equation*}
        \begin{split}
            \sin \theta &= \sqrt{1 - \cos^2 \theta}\\
            &= \sqrt{1 - \left(\mathbf{\frac{ab}{|a||b|}}\right)^2}\\
            &= \sqrt{\frac{(a_1^2 + a_2^2 + a_3^2)(b_1^2 + b_2^2 + b_3^2) - (a_1b_1 + a_2b_2 + a_3b_3)^2}
            {(a_1^2 + a_2^2 + a_3^2)(b_1^2 + b_2^2 + b_3^2)}}\\
            &= \sqrt{\frac{a_1^2b_2^2 +a_1^2b_3^2 + a_2^2b_1^2 + a_2^2b_3^2 + a_3^2b_1^2 + a_3^2b_2^2 - 2a_1a_2b_1b_2 - 2a_1a_3b_1b_3 - 2a_2a_3b_2b_3}
            {\mathbf{(|a||b|)^2}}}\\
            &= \sqrt{\frac{(a_1b_2-a_2b_1)^2 + (a_1b_3-a_3b_1)^2 + (a_2b_3-a_3b_2)^2}
            {\mathbf{(|a||b|)^2}}}
        \end{split}
    \end{equation*}
    i.e. Area of parallelogram $= \mathbf{|a||b|}\sin \theta = |\mathbf{a\times b}|$
\end{addmargin}
\end{document}