\documentclass{article}
\usepackage[utf8]{inputenc}
\usepackage{amsmath}
\usepackage{scrextend}
\usepackage{setspace}
\usepackage{amsfonts}
\usepackage{amssymb}
\usepackage{graphicx}
\graphicspath{ {../../images/} }

\title{Calculus\\
\large{Notes on 2023/03/28}}
\author{shaozewxy }
\date{March 2023}

\doublespacing
\begin{document}

\maketitle

\section{Geometry of $\mathbb{R}^n$ (Ch 1.4)}
\textbf{Algebra is all about eqaulity, calculus is all about inequality.}
\subsection{Interpretation of dot product}

\begin{equation*}
    \mathbf{x} \cdot \mathbf{y} = |\mathbf{x}||\mathbf{y}| \cos(\theta) \tag{1.4.3}
\end{equation*}
Proof:
\begin{addmargin}[10px]{0px}
    This comes from the \textbf{cosine law}.\\
    Since we know $\mathbf{x}$, $\mathbf{y}$, $\mathbf{x}-\mathbf{y}$ form a triangle, therefore:
    \begin{equation*}
        |\mathbf{x} - \mathbf{y}|^2 = |\mathbf{x}|^2 + |\mathbf{y}|^2 - 2 |\mathbf{x}| |\mathbf{y}| \cos \alpha
    \end{equation*}
    Since also,
    \begin{equation*}
        |\mathbf{x} - \mathbf{y}|^2 = (\mathbf{x} - \mathbf{y})^2
    \end{equation*}
    We have
    \begin{equation*}
        \begin{split}
            |\mathbf{x}|^2 + |\mathbf{y}|^2 - 2 \mathbf{x}\mathbf{y} &= |\mathbf{x}|^2 + |\mathbf{y}|^2 - 2 |\mathbf{x}| |\mathbf{y}| \cos \alpha\\
            \mathbf{x}\mathbf{y} &= |\mathbf{x}||\mathbf{y}|\cos \alpha
        \end{split}
    \end{equation*}
\end{addmargin}
\subsection{Angle between vectors in $\mathbb{R}^n > 3$}
We want to define angles between $\mathbf{x}, \mathbf{y} \in \mathbb{R}^n, n > 3$ with
\begin{equation*}
    \alpha = \arccos\frac{\mathbf{x}\mathbf{y}}{|\mathbf{x}||\mathbf{y}|}
\end{equation*}
However we cannot guarantee that $\frac{\mathbf{x}\mathbf{y}}{|\mathbf{x}||\mathbf{y}|} \in [-1, 1]$. Hence we need to first prove the following:\\
\textbf{Schwartz's Inequality}
\begin{equation*}
    \forall \mathbf{v}, \mathbf{w} \in \mathbb{R}^n, |\mathbf{v}\mathbf{w}| \leq |\mathbf{v}||\mathbf{w}| \tag{1.4.5}
\end{equation*}
\textbf{Proof:}
\begin{addmargin}[10px]{0px}
    Suppose $\mathbf{v}, \mathbf{w} \neq 0$, then consider $f(t) = |\mathbf{v} + t \mathbf{w}|^2$:
    \begin{equation*}
        \begin{split}
            f(t) = |\mathbf{w}|^2 t^2 + 2\mathbf{v}\mathbf{w} t + |\mathbf{v}|^2
        \end{split}
    \end{equation*}
    It is obvious that $f(t) \geq 0$, this means that
    \begin{equation*}
        \begin{split}
            (2\mathbf{v}\mathbf{w})^2 - 4|\mathbf{v}|^2|\mathbf{w}|^2 &< 0\\
            4|\mathbf{v}\mathbf{w}|^2 &< 4(|\mathbf{v}||\mathbf{w}|)^2
        \end{split}
    \end{equation*}
\end{addmargin}
With Schwartz's Inequality, we have proven that
\begin{equation*}
\forall \mathbf{v}, \mathbf{w} \in \mathbb{R}^n, -1 \leq \frac{\mathbf{vw}}{|\mathbf{v}||\mathbf{w}|} \leq 1
\end{equation*}
And now we can define the angle between any two vectors as:
\begin{equation*}
    \alpha = \arccos \frac{\mathbf{vw}}{|\mathbf{v}||\mathbf{w}|}
\end{equation*}
\subsubsection{Application of Schwartz's Inequality}
\begin{equation*}
    \forall \mathbf{x, y} \in \mathbb{R}^n, |\mathbf{x+y}| \leq |\mathbf{x}| + |\mathbf{y}| \tag{1.4.9}
\end{equation*}
\textbf{Proof:}
\begin{addmargin}[10px]{0px}
    For the LHS, we have
    \begin{equation*}
       \begin{split}
            |\mathbf{x+y}|^2 &= (\mathbf{x+y})^2\\
            &= |\mathbf{x}|^2 + |\mathbf{y}|^2 + 2\mathbf{xy}
       \end{split} 
    \end{equation*} 
    For the RHS, we have
    \begin{equation*}
        (|\mathbf{x}| + |\mathbf{y}|)^2 = |\mathbf{x}|^2 + |\mathbf{y}|^2 + 2\mathbf{|x||y|}
    \end{equation*}
    Since from Schwartz's Inequality we know that $2\mathbf{xy} \leq 2|x||y|$, therefore,
    \begin{equation*}
        |\mathbf{x+y}|^2 = |\mathbf{x}|^2 + |\mathbf{y}|^2 + 2 \mathbf{xy} \leq (\mathbf{|x| + |y|})^2 = |\mathbf{x}|^2 + |\mathbf{y}|^2 + 2 |\mathbf{x}||\mathbf{y}|
    \end{equation*}
\end{addmargin}
\subsection{Geometry of Matrices}
First we need to define \textbf{the length of a matrix:}
\begin{equation*}
    \forall A \in \mathbb{R}^{n\times m}, |A|^2 = \sum_{i,j} a_{ij}^2 \tag{1.4.10}
\end{equation*}
Then we can define the \textbf{product involving matrices:}
\begin{equation*}
    \forall A \in \mathbb{R}^{n\times m}, B \in \mathbb{R}^{m\times k}, b \in \mathbb{R}^m |A\mathbf{b}| \leq |A| |\mathbf{b}|, |AB| \leq |A| |B| \tag{1.4.11}
\end{equation*}
\textbf{Proof:}
\begin{addmargin}[10px]{0px}
    Using $A_{i\_}$ to denote the vector consisting of $[a_{i1}, a_{i2}, ..., a_{im}]$, we have:
    \begin{equation*}
        |A\mathbf{b}|^2 = \sum_{1\leq i \leq n} (A_{i\_}\mathbf{b})^2
    \end{equation*}
    while
    \begin{equation*}
        \begin{split}
            (|A||b|)^2 &= \sum_{1\leq i\leq n} |A_{i\_}|^2 |\mathbf{b}|^2
        \end{split}
    \end{equation*}
    Since from Schwartz's Inequality we know that $(A_{i\_}b)^2 \leq |A_{i\_}|^2|\mathbf{b}|^2$, we have proven that
    \begin{equation*}
        |A\mathbf{b}|^2 \leq (|A||\mathbf{b}|)^2
    \end{equation*}
\end{addmargin}
For $|AB|^2$, we know that
\begin{equation*}
    |AB|^2 = \sum_{1\leq j\leq k} |AB_{\_j}|^2
\end{equation*}        
And since for each $|AB_{\_j}|^2 \leq (|A||B_{\_j}|)^2$ (proven previously), we know that
\begin{equation*}
    |A|^2 |B|^2 = |A|^2 \sum_{1\leq j\leq k} B_{\_j}^2 \geq |AB|^2
\end{equation*}
\end{document}