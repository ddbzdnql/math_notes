\documentclass{article}
\usepackage[utf8]{inputenc}
\usepackage{amsmath}
\usepackage{scrextend}
\usepackage{setspace}
\usepackage{amsfonts}
\usepackage{amssymb}
\usepackage{braket}

\title{Week 4 Notes\\
\large{Abstract Algebra}}
\author{shaozewxy }
\date{May 2022}

\doublespacing
\begin{document}
\maketitle
\setcounter{secnumdepth}{0}
\section{3.1 Quotient Groups}
\textbf{Definition of quotient group:}\\
Given $\phi: G \rightarrow H$ a homomorphism with kernel $K$. The quotient group $G/K$ is the group whose elements are fibres of $\phi$. The group operation is defined as $\Bar{a} \cdot \Bar{b} = \Bar{ab}$.\\
\textbf{Definition of coset:}\\
Given $N \leq G, g \in G$, define
\begin{equation*}
    gN = \{gn|n \in N\}
\end{equation*}
$Ng$ is similarly defined. These are called a left coset and a right coset.\\
\textbf{Cosets form a group:}\\
$G$ a group and $K$ the kernel of some homomorphism from $G$. Then the cosets of $K$ in G form a group $G/K$ with the group operation defined as
\begin{equation*}
    uK \cdot vK = (uv)K
\end{equation*}
\textbf{Proof:}\\
\begin{addmargin}[1em]{0em}
NTS the group operation is well define, i.e. $\forall u_1K = u_2K, v_1K = v_2K, (u_1v_1)K = (u_2v_2)K$:\\
Because $K$ a kernel of $\phi: G$ and $u_1K = u_2K$, then $\phi(u_1) = \phi(u_2)$, similarly $\phi(v_1) = \phi(v_2)$.\\
Then $k \in K, \phi(u_1v_1k) = \phi(u_1)\phi(v_1)\phi(k) = \phi(u_2)\phi(v_2) \rightarrow u_1v_1k \in \phi^{-1}(\phi(u_2v_2)) = u_2v_2K$.\\
Therefore the group operation is well-defined.
\end{addmargin}
We then formally define the condition for which the cosets of a subgroup forms a group:\\
\textbf{Cosets of subgroup partitions the group:}\\
Given $N \leq G$, set of left cosets form a partition of $G$. And $\forall u, v \in G, uN = vN \iff v^{-1}u \in N$.\\
\textbf{Proof:}
\begin{addmargin}[1em]{0em}
First it is clear that the cosets contains $G$. We only NTS that $\forall uN \neq vN, uN \cap vN = \varnothing$.\\
Suppose $x \in uN \cap vN$, then $x = ua = vb, a, b \in N$.\\
Therefore $u = vba^{-1} \in vN \rightarrow \forall un \in uN, un = vba^{-1}n \in vN$, therefore $uN \subseteq vN$. Similarly $vN \subseteq uN$, i.e. $uN = vN$.\\
This proves that left cosets of $N$ partitions $G$. It also proves the second statement.
\end{addmargin}
\textbf{Conditions for multiplication of cosets:}\\
The operationj $uN \cdot vN = (uv)N$ is well defined iff
\begin{equation*}
    \forall g \in G, n \in N, gng^{-1} \in N
\end{equation*}
\textbf{Proof:}
\begin{addmargin}[1em]{0em}
Suppose $uN \cdot vN = (uv)N$ is well defined, then NTS $\forall g \in G, n \in N, gng^{-1} \in N$:\\
We have $gN \cdot g^{-1}N = N \rightarrow gn \in gN, g^{-1}n \in g^{-1}N, \exists n' \in N, gng^{-1}n = n' \rightarrow gng^{-1} = n'n^{-1} \in N$.\\
Suppose $\forall g \in G, n \in N, gng^{-1} \in N$, then NTS $uN \cdot vN$ is well defined:\\
Given $u, u_1, uN = u_1N, v, v_1, vN = v_1N$, NTS $uv \in (u_1v_1)N$.\\
By definition, $\exists a, b \in N, u_1 = ua, v_1 = vb$, then $u_1v_1 = uavb = u(vv^{-1})avb = uv(n')b$ where $n' = v^{-1}av \in N$, therefore $u_1v_1 \in uvN$. Therefore $u_1v_1N = uvN$.
\end{addmargin}
Similar to the case of the kernel, we say that as long as the operation of multiplication is well defined, then the set of cosets is a group.\\
We can generalize some definitions from the above theorem:\\
$gng^{-1}$ is called the \textbf{conjugate} of $n \in N$ by $g$.\\
$gNg^{-1}$ is similarly defined.\\
We say $g$ \textbf{normalizes} $N$ if $gNg^{-1} = N$.\\
Subgroup $N$ of $G$ is \textbf{normal} if every element in $G$ normalizes $N$, and we write $N \trianglelefteq G$.\\
\textbf{Various presentations of normality:}\\
Given $N$ a subgroup of $G$, the following conditions are equivalent:
\begin{addmargin}[1em]{0em}
$N \trianglelefteq G$\\
$N_G(N) = G$\\
$\forall g \in G, gN = Ng$\\
Left cosets of $N$ form a group.\\
$\forall g \in G, gNg^{-1} \subseteq N$
\end{addmargin}
\textbf{Relationship between kernels and normal subgroups:}\\
$N \trianglelefteq G \iff N$ is a kernel of some homomorphism.\\
\textbf{Proof:}\\
\begin{addmargin}[1em]{0em}
Suppose $N \trianglelefteq G$, NTS $N$ kernel of some homomorphism.\\
We create $\phi: G \rightarrow G/N$ by
\begin{equation*}
    \phi(g) = gN
\end{equation*}
Because $N \trianglelefteq G$, we know $G/N$ is a group and therefore $\phi$ is a homomorphism. NTS $ker\ \phi = N$.\\
$\forall g \in N, \phi(g) = gN = N$, therefore $N \subseteq ker\ \phi$.\\
$\forall g \in ker\ \phi, gN = N \rightarrow g \in N$, therefore $ker\ \phi \subseteq N$.\\
Therefore $ker\ \phi = N$.\\
Similarly if $N$ is a kernel of some homomorphism. Then the left cosets of $N$ is a group and therefore $G \trianglelefteq N$.
\end{addmargin}
We call the homomorphism defined above \textbf{natural projection}\
\section{3.2 More on Cosets}
\textbf{Lagrange's Theorem:}\\
$G$ a finite group and $H \leq G$, the $|H|||G|$, i.e. the order of $H$ divides order of $G$. Moreover, the number of left cosets of $H$ in $G$ is $\frac{|G|}{|H|}$.\\
\textbf{Proof:}
\begin{addmargin}[1em]{0em}
First we know that the left cosets partitions $G$ and we onlt NTS $\forall g \in G, |gH| = |H|$:\\
The map $\phi: H \rightarrow gH$ defined by:
\begin{equation*}
    \phi(h) = gh
\end{equation*}
is by definition surjective. And because the left cancellation rule, we have $\forall h_1, h_2 \in H, gh_1 = gh_2 \rightarrow h_1 = h_2$, therefore $\phi$ is injective.\\
Therefore $\phi$ is a bijection and $|gH| = |H|$.\\
Therefore we have
\begin{equation*}
    |G| = \sum_{\textrm{distinct }gH} |gH| = \sum_{\textrm{number of left cosets}} |H|
\end{equation*}
\end{addmargin}
\textbf{Consequences of Lagrange's Theorem:}\\
$G$ a finite group. $\forall x \in G, |x|$ divides $|G|$ and $x^{|G|} = 1$.\\
$|G|$ a prime $\rightarrow$ $G$ cyclic and therefore $G \cong Z_p$.\\
\textbf{Proof:}
\begin{addmargin}[1em]{0em}
$|x| = |\braket{x}|$ and therefore $|x|$ divides $|G|$.\\
Because $|G|$ a prime, then $\exists x \neq 1 \in G$, and $|\braket{x}|$ divides $|G| \rightarrow |\braket{x}| = |g| \rightarrow \braket{x} = G$
\end{addmargin}
\subsection{Examples}
\textbf{1. $\braket{(1\ 2\ 3)} \trianglelefteq S_3$:}
\begin{addmargin}[1em]{0em}
$H = \braket{(1\ 2\ 3)} \leq S_3 = G$. We know that
\begin{equation*}
    \braket{H} \leq N_G(H) \leq G
\end{equation*}
Since $|\braket{H}| = 3, |G| = 6$ we know $|N_G(H)| = |\braket{H}|$ or $|G|$.\\
Now since $(1\ 2)(1\ 2\ 3)(1\ 2)^{-1} = (1\ 3\ 2) = (1\ 2\ 3)^{-1}$, we know that $|N_G(H)| \neq |\braket{H}| \rightarrow |N_G(H)| = |G| \rightarrow N_G(H) = G \rightarrow H \trianglelefteq G$.
\end{addmargin}
\textbf{2. Subgroups of index $2$ are normal:}
\begin{addmargin}[1em]{0em}
Given $H \leq G, |G:H| = 2$, then there are only 2 left cosets: $H, gH, g \notin H$.\\
Because left cosets partition $G$, we know that $gH = G - H$.\\
Similarly, there are only two right cosets: $H, Hg$, for similar reasons $Hg = G-H = gH$, therefore $H \trianglelefteq G$.
\end{addmargin}
\textbf{3. Subgroup of fixed-point permutations are not normal:}\\
Let $G = S_n$, and for some $i \in \{1,2, ..., n\}$, define
\begin{equation*}
    G_i = \{\sigma \in G| \sigma(i) = i\}
\end{equation*}
i.e. subgroup of permutations that fixes $i$. This group is not normal.\\
\textbf{Proof:}
\begin{addmargin}[1em]{0em}
Create $\tau \in G, \tau(i) = j$, we calculate $\tau G_i$:\\
$\forall \mu \in G_i, \tau(\mu (i)) = \tau(i) = j$, therefore
\begin{equation*}
    \tau G_i = \{\tau \in G| \tau(i) = j\}
\end{equation*}
For the same $\tau$, we say that $\tau(k) = i$, and calculate $G\tau$:\\
$\forall \mu \in G_i, \mu(\tau(k)) = \mu(i) = i$, therefore
\begin{equation*}
    G_i\tau = \{\tau \in G| \tau(k) = i\}
\end{equation*}
Therefore we can definitely find $\tau \in G$ that doesn't satisfy both conditions at the same time.
\end{addmargin}
\textbf{4. Converse of Lagrange's Theorem is not true:}
\begin{addmargin}[1em]{0em}
Let $A$ = the group of symmetries of a tetrahedron and $|A| = 12$. So suppose $\exists H \leq A, |H| = 6$, this means there are only 2 left cosets: $H, gH$. Since $|G:H| = 2 \rightarrow H \trianglelefteq G$ and $A/H \cong Z_2$, so we have $(gH)^2 = H \rightarrow \forall g \in G, g^2 \in H$.\\
Therefore, $\forall g \in G, |g| = 3, g = g^3g = (g^2)^2 \in H$, i.e. all elements of order $3$ are in $H$, but there are at least $8$ rotations of order $3$ ($2$ for each vertex). So there is no subgroup of order $6$ in $A$ although $|A| = 12$.
\end{addmargin}
\subsection{Partial converses of Lagrange's Theorem:}
\textbf{Cauchy's Theorem:}\\
$G$ a finite group, and $p$ a prime that divides $|G|$, then $G$ has an element of order $p$.\\
\textbf{Sylow's Theorem:}\\
$G$ a finite group of oder $p^\alpha m$ with $p$ a prime and $p$ doesn't divide $m$, then $G$ has a subgroup of order $p^\alpha$.\\
\subsection{Results related to Cosets}
$H, H \leq G$, then
\begin{equation*}
    |HK| = \frac{|H||K|}{|H\cap K|}
\end{equation*}
\textbf{Proof:}
\begin{addmargin}[1em]{0em}
It is clear that
\begin{equation*}
    HK = \bigcap_{\textrm{distinct }h} hK
\end{equation*}
Therefore we only need to find number of distinct $hK$s. We claim that given $h_1 \in H, \{h \in H | hK = h_1K\} = h_1(H \cap K)$:\\
Suppose $h \in H, hK = h_1K$, then this means $\exists k_1 \in K,  h = h_1k_1 \rightarrow h \in h_1(H\cap K)$, therefore $\{h \in H | hK = h_1K\} \subseteq h_1(H \cap K)$.\\
Suppose $h = h_1k_1 \in h_1(H \cap K)$, then $hK = h_1k_1K = h_1K$. Therefore $h_1(H \cap K) = \{h \in H | hK = h_1K\}$.\\
Therefore there are $|H:H \cap K|$ distinct $hK$, i.e. $|HK| = \frac{|H|}{|H \cap K|} \cdot |hK| = \frac{|H||K|}{|H \cap K|}$.
\end{addmargin}
\section{3.3 Isomorphism Theorems}
\subsection{Fundamental Theorem of Homomorphism:}
If $\phi:G \rightarrow H$ is a homomorphism, then $\ker\ \phi \trianglelefteq G$ and $G/ker\ \phi \cong \phi(G)$.\\
This is just restating result from 3.1.
\subsection{Second (Diamond) Isomorphism Theorem:}
$G$ a group, $A, B \leq G$ and $A \leq N_G(B)$. Then:\\
$AB \leq G, B \trianglelefteq AB, A \cap B \trianglelefteq A, AB/B \cong A/A \cap B$.\\
\textbf{Proof:}
\begin{addmargin}[1em]{0em}
$AB \leq G$ is easy to prove, just use the fact that $A$ normalizes $B$.\\
For the rest two, consider $\phi: A \rightarrow AB/B$ defined by
\begin{equation*}
    \phi(a) = aB
\end{equation*}
Because $AB$ is a group, then $AB/B$ is also a group, therefore $\phi$ is a homomorphism.\\
Here $ker\ \phi = A \cap B$ obviously and therefore from the First Isomorphism Theorem, we have:\\
$A/A\cap B \cong AB/B$ and $A\cap B \trianglelefteq A$
\end{addmargin}
\subsection{Third Isomorphism Theorem}
$G$ a group and $H, K \trianglelefteq G, H \leq K$, then\\
$K/H \trianglelefteq G/H, (G/H)/(K/H) \cong (G/K)$\\
\textbf{Proof:}
\begin{addmargin}[1em]{0em}
Define $\phi: G/H \rightarrow G/K$ by
\begin{equation*}
    \phi(gH) = gK
\end{equation*}
This is well defined $H \leq K$.\\
$ker\ \phi = \{gH | \phi(gH) = K\} = \{gH | gK = K\} = \{gH | g \in K\} = K/H$.\\
Therefore we have $K/H \trianglelefteq G/H$ and $(G/H)/(K/H) \cong G/K$
\end{addmargin}
\section{The Fourth Isomorphism Theorem}
$G$ a group and $N \trianglelefteq G$, then there is a bijection from the set of subgroups $A$ of $G$ that contains $N$ onto the set of subgroups $\Bar{A} = A/N$ of $G/N$.\\
This means that every subgroup of $G/N$ is of the form $A/N$ for some $A \leq G$ that contains $N$.\\
This bijection has the following properties, given $A, B \leq G, N \leq A, N \leq B$:\\
$A \leq B \iff \overline{A} \leq \overline{B}$\\
$A \leq B \rightarrow |B:A| = |\overline{B}:\overline{A}|$\\
$\overline{\braket{A, B}} = \braket{\overline{A}, \overline{B}}$\\
$\overline{A \cap B} = \overline{A} \cap \overline{B}$\\
$A \trianglelefteq G \iff \overline{A} \trianglelefteq \overline{G}$
\section{3.5 Transposition and Alternating Group}
We call a $2-cycle$ a \textbf{transposition}.\\
It is clear that $\forall \sigma \in S_n, \sigma$ can be written as a product of transpositions\\
The reason is that every cycle $(a_1\ a_2\ ...\ a_m)$ can be written as $(a1\ a_m)(a_1\ a_{m-1})...(a_1\ a_2)$, and every $\sigma$ can be written as product of cycles.\\
\subsection{Alternating Group}
While there are different ways to write $\sigma \in S_n$ as product of transpositions, the parity, i.e. the odd/even-ness of the transpositions is the same for a fixed $\sigma$.\\
\textbf{Parity of Permutations:}\\
Given $x_1, ..., x_n$, we denote $\Delta$ as
\begin{equation*}
    \Delta = \prod_{1\leq i < j \leq n} (x_i - x_j)
\end{equation*}
i.e. product of $(x_i - x_j)$ for all $j > i$.\\
$\forall \sigma \in S_n$, let $\sigma$ act on $\Delta$ by permuting the indices of the variables, i.e
\begin{equation*}
    \sigma(\Delta) = \prod_{1\leq i < j \leq n} (x_{\sigma(i)} - x_{\sigma(j)})
\end{equation*}
Originally $\forall i < j, (x_i - x_j) \in \Delta$. After applying $\sigma$, some $(x_i-x_j)$ becomes $(x_j - x_i) = -(x_i - x_j)$. Therefore, $\sigma(\Delta) = \pm \Delta$.\\
We then define
\begin{equation*}
    \epsilon(\sigma) = \begin{cases}
    \plus 1 & \sigma(\Delta) = \Delta \\
    -1 & \sigma(\Delta) = -\Delta
    \end{cases}
\end{equation*}
We then define $\epsilon(\sigma)$ the \textbf{sign} of $\sigma$, and from the sign, define \textbf{even/odd} of $\sigma$.
\textbf{Sign of Permutation is Homomorphic}:\\
The map $\epsilon:S_n \rightarrow \{\pm 1\}$ defines a homomorphism where $\{\pm 1\}$ is a multiplicative group.\\
\textbf{Proof:}
\begin{addmargin}[1em]{0em}
Given $\tau, \sigma \in S_n$, then $(\tau\sigma)(\Delta) = \prod_{1 \leq i < j \leq n}(x_{\tau\sigma(i)} - x_{\tau\sigma(j)})$.\\
Suppose $\sigma(\Delta) = (-1)^k\Delta$, then $\tau\sigma(\Delta) = (-1)^k(-1)^m\Delta$.\\
$\epsilon(\tau\sigma) = (-1)^{k+m} = (-1)^k(-1)^m = \epsilon(\tau)\epsilon(\sigma)$.
\end{addmargin}
\textbf{Transpositions are Odd:}\\
\textbf{Proof:}
\begin{addmargin}[1em]{0em}
Given $i < j$,
$\forall k < i, (x_k - x_i)$ becomes $(x_k - x_i)$ which produces no changes.\\
$\forall k > j, (x_j - x_k)$ becomes $(x_i - x_k)$ which produces no changes.\\
$\forall i < k < j, (x_i - x_k)$ becomes $(x_j - x_k)$ while $(x_k - x_j)$ becomes $(x_k - x_i)$, which produces $(-1)^2 = 1$, equivalent to no changes.\\
Then for $(x_i - x_j)$ becomes $(x_j - x_i)$ produces $-1$.\\
Therefore $\epsilon((i\ j)) = -1$.
\end{addmargin}
Therefore we can see that the parity of number of transpositions for a fixed $\sigma \in S_n$ will always stay the same.
\end{document}