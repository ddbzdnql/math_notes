\documentclass{article}
\usepackage[utf8]{inputenc}
\usepackage{amsmath}
\usepackage{scrextend}
\usepackage{setspace}
\usepackage{amsfonts}
\usepackage{braket}
\usepackage{amssymb}
\usepackage{graphicx}
\graphicspath{ {../../images/} }

\title{Abstract Algebra\\
\large{Week 3 Notes (b)}}
\author{shaozewxy }
\date{September 2022}

\doublespacing
\begin{document}
\maketitle
\section*{Quotient Groups: Definitions and Examples}
The study of \textbf{quotient groups} are essentially the study of \textbf{fibres} of $\phi$ i.e. the set of elements mapping to the same element under $\phi$.
\subsection*{Basic properties of homomorphisms and fibres}
\subsubsection*{Definition of quotient group}
Given $\phi: G \rightarrow H$ be a homomorphism with kernel $K$. The \textbf{quptient group G/K}  is the group whose element elements are fibres of $\phi$ with operation defined as:
\begin{equation*}
    X = \overline{a}, Y = \overline{b}, XY = \overline{ab}
\end{equation*}
We can also define a quotient group without spelling out the $\phi$ explicitly.
\subsubsection*{Properties of fibres and kernels}
Given $\phi: G \rightarrow H$ a homomorphism with kernel $K$. Suppose $X \in G/K$ is a fibre above $a$, i.e. $X = \phi^{-1}(a)$. Then
\begin{enumerate}
    \item $\forall u \in X, X = \{uk| k \in K\}$.
    \item $\forall u \in X, X = \{ku| k \in K\}$.
\end{enumerate}
This is saying that if $K$ is a kernel, then $uK = Ku$.
Proof for this is obvious.\\
The groups defined above can be generalized to any subgroup in $G$:
\subsubsection*{Definition of coset}
Given $N \leq G, g \in G$, let
\begin{equation*}
    gN = \{gn | n \in N\}, Ng = \{ng | n \in N\}
\end{equation*}
these are called \textbf{left coset and right coset} respectively. Any element of a coset is called a \textbf{representative} of that coset.\\
With the definiton, we can see that \textbf{fibres of homomorphisms} are just \textbf{left cosets of kernels}.\\
This enables us to define quotient groups without spelling out the homomorphisms explicitly as long as we know that $K$ is a kernel:
\subsubsection*{Operations of cosets}
Given $G$ a group and $K$ a kernel of some homomorphism from $G$. Then the set of left cosets of $K$ with operations defined by
\begin{equation*}
    uK \circ vK = (uv)K
\end{equation*}
forms a group $G/K$.\\
\textbf{Proof:}
\begin{addmargin}[1em]{0em}
    Given $X, Y \in G/K$ with $X = \phi^{-1}(a), Y = \phi^{-1}(b)$, we denote that $Z = XY = \phi^{-1}(ab)$. We NTS that $\forall u \in X, v \in Y, uv \in Z$:\\
    Since $u \in X, v \in Y$, we know that $\phi(u) = a, \phi(v) = b$. Then $\phi(uv) = \phi(u)\phi(v) = ab \rightarrow uv \in Z = \phi^{-1}(ab)$.\\
    Also NTS show that $\forall z = uvk \in Z = (uv)K, \exists u' \in X, v' \in Y$ such that $z = u'v'$:\\
    Since $z \in Z$, we know that $\phi(z) = ab$. Then choose $u \in X = uK, z = u \circ (u^{-1}z)$, we have\\
    $\phi(u)\phi(u^{-1}z) = a \circ \phi(u^{-1}z) = ab \rightarrow \phi(u^{-1}z) = b \rightarrow u^{-1}z \in Y$.\\
    THerefore we have shown that $Z = uvK = XY = (uK)(vK)$.
\end{addmargin}
Here we note that the multiplication of cosets is \textbf{independent of the choice of representations}.\\
We can therefore denote a coset $X = uK = \overline{u}$, then multiplication can just be written as $\overline{uv}$.\\
Next we show that $G/K$ is well defined $\iff K$ is a kernel. Such a subgroup $K$ is also called a \textbf{normal subgroup}.
\subsubsection*{Cosets form a partition}
Given $N \leq G$, the set of left cosets of $N$ form a partition of $G$. Furthermore $\forall u, v \in G, uN = vN \iff v^{-1}n \in N$.\\
\textbf{Proof:}
\begin{addmargin}[1em]{0em}
    First $\forall g \in G, g \in gN$ since $1 \in N$. Therefore
    \begin{equation*}
        G = \bigcup_{g \in G} gN
    \end{equation*}
    Then NTS that $uN \neq vN \rightarrow uN \cap vN = \emptyset$:\\
    Suppose $\exists x \in uN, vN$, we claim that $uN = vN$:\\
    Denote $x = un = vm$ for some $n, m \in N$, then we have $u = vmn^{-1}$ with $mn^{-1} \in N$.\\
    Therefore $\forall ut \in uN, ut = vmn^{-1}t \in vN$, i.e. $uN \subseteq vN$.\\
    We can similarly prove $vN \subseteq uN$.\\
    Therefore $uN = vN$.
\end{addmargin}
\subsubsection*{Conditions for coset multiplication}
Given $N \leq G$.
\begin{enumerate}
    \item The multiplication
    \begin{equation*}
        uN \cdot vN = (uv)N
    \end{equation*}
    is well defined $\iff \forall g \in G, n \in N, gng^{-1} \in N$.
    \item If the operation is well defined, then left cosets of $N$ is a group. With identity $1N$ and $(gN)^{-1} = g^{-1}N$.
\end{enumerate}
\textbf{Proof:}
\begin{addmargin}[1em]{0em}
    Suppose $\forall g \in G, n \in N, gng^{-1} \in N$, WTS $uN\cdot vN = (uv)N$ is well defined:\\
    Given $u' = un \in uN, v' = vm \in vN$, NTS $u'v' \in (uv)N$:\\
    $u'v' = unvm = uv\cdot v^{-1}nvm$. Since $\forall g \in G, n \in N, gng^{-1} \in N$, we know that $\exists n' \in N$ such that $n' = v^{-1}nv$.\\
    Therefore $u'v' = uv\cdot n'm \in (uv)N$.\\
    Then suppose $uN \cdot vN$ well defined, WTS $\forall g \in G, n \in N, gng^{-1} \in N$:\\
    Here given $g \in G$, we have $gN \cdot (g^{-1}N) = (gg^{-1})N = 1N$.\\
    This means that $\forall n \in N$, since $gn \in gN, g^{-1} \in g^{-1}N$, we know that $gng^{-1} \in N$, i.e. $\exists n' \in N$ such that $1\cdot n' = n' = gng^{-1}$.\\
    This complete the proof for 1. The proof for 2 is easy.
\end{addmargin}
\subsubsection*{Definition of conjugate and normal}
The element $gng^{-1}$ is called the \textbf{conjugate} of $n \in N$ by $g$.\\
The set $gNg^{-1} = \{gng^{-1} | n \in N\}$ is called the \textbf{conjugate} of $N$ by $g$.\\
The element $g$ is said to \textbf{normalize} $N$ if $gNg^{-1} = N$.\\
$N \leq G$ is calle \textbf{normal} if every element normalizes $N$, written $N \trianglelefteq G$.\\
Note that the structure of $G$ is preserved in $G/N$.
\subsubsection*{Conditions for normal subgroups}
Given $N \leq G$. The following are equivalent:
\begin{enumerate}
    \item $N \trianglelefteq G$
    \item $N_G(N) = G$
    \item $\forall g \in G, gN = Ng$
    \item The coset multiplication for $N$ make the cosets into a group.
    \item $\forall g \in G, gNg^{-1} \subseteq N$.
\end{enumerate}
Proof of this is done throughout this chapter.\\
In determining the normality of $N$, using generators can avoid a lot of computations since proving \textbf{normality for each of the generators suffices to show normality of the whole group}.\\
Now we prove the relation between kernel and normal subgroups:
\subsubsection*{Kernel and normal subgroups}
$N \leq G$ is normal $\iff$ it is a kernel of some homomorphism.\\
\textbf{Proof:}
\begin{addmargin}[1em]{0em}
    Suppose $N$ is a kernel of $\phi$, WTS $N \trianglelefteq G$:\\
    $\forall g \in G, \phi(gng^{-1}) = \phi(g)\phi(n)\phi(g^{-1}) = \phi(g)\phi(g^{-1}) = 1 \rightarrow gng^{-1} \in N$ the kernel of $\phi$.\\
    Therefore $N_G(N) = G \rightarrow N \trianglelefteq G$.\\
    Suppose $N \trianglelefteq G$, WTS $\exists \phi$ such that $N$ is the kernel of $\phi$:\\
    Just use
    \begin{equation*}
        \forall g \in G, \phi(g) = gN
    \end{equation*}
\end{addmargin}
The homomorphism defined above is special:
\subsubsection*{Definition of natural projection}
Given $N \trianglelefteq G$, then homomorphism $\pi: G \rightarrow G/N$ defined by $\pi(g) = gN$ is called the \textbf{natural projection} of $G$ onto $G/N$.\\
The normalizer of a subgroup $N \leq G$ is a measure of how close $N$ is to being normal.
\end{document}