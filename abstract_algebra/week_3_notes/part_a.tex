\documentclass{article}
\usepackage[utf8]{inputenc}
\usepackage{amsmath}
\usepackage{scrextend}
\usepackage{setspace}
\usepackage{amsfonts}
\usepackage{braket}
\usepackage{amssymb}
\usepackage{graphicx}
\graphicspath{ {../../images/} }

\title{Abstract Algebra\\
\large{Week 3 Notes (a)}}
\author{shaozewxy }
\date{September 2022}

\doublespacing
\begin{document}
\maketitle
\section*{2.3 Cyclic Groups and Cyclic Subgroups}
\subsection*{Definition of cyclic group}
A group $H$ is \textbf{cyclic} if $H$ can be generated by a single element, i.e., $\exists x \in H$ such that $H = \{x^n|n \in \mathbb{Z}\}$\\
In this case we can write $H = \braket{x}$.\\
\textbf{A cyclic group might have more than 1 generators:}\\
Given cyclic group $H = \braket{x}$, then $H = \braket{x^{-1}}$ since $\forall x^n, x^n = (x^{-1})^{-n}$.
\subsubsection*{Examples}
\begin{enumerate}
    \item Given $G = D_{2n} = \braket{r, s|r^n=s^2=1, rs=sr^{-1}}, n \geq 3$. And let $H$ be the subgroup of all rotations.\\
    Then $H = \braket{r} = \{1, r, r^2, ..., r^{n-1}\}, |H| = |r| = n$.\\
    Therefore we can write any $r^t$ as $r^k$ where
    \begin{equation*}
        t = nq + k \tag*{$0 \leq k \leq n$}
    \end{equation*}
    For example in $D_8, r^4 = 1 \rightarrow r^{105} = r, r^{-42} = r^{4(-11)+2} = r^2$.
    \item Given $H = \mathbb{Z}$, then $H = \braket{1}$ with operation $+$.
\end{enumerate}
\subsection*{Properties of cyclic groups}
\subsubsection*{Composition of cyclic groups}
Given $H = \braket{x}$, then $|H| = \braket{x}$ where if one side is infinite, then the other is also infinite.
\begin{enumerate}
    \item If $|H| = n < \infty \rightarrow x^n = 1, H = \{1, x, x^2, ..., x^{n-1}\}$.
    \item If $|H| = \infty \rightarrow \forall n \neq 0, x^n \neq 1$ and $\forall a\neq b \in \mathbb{Z}, x^a \neq x^b$.
\end{enumerate}
\textbf{Proof:}
\begin{addmargin}[1em]{0em}
    For 1, let $|x| = n$, then clearly $1, x, x^2, ..., x^{n-1}$ are distinct since if $\exists a, b < n, x^a = x^b \rightarrow x^{b-a} = 1$ contrary to the fact that $n$ is the smallest integer such that $x^n = 1$.\\
    Thus $H$ has at least $n$ elements, then only NTS $|H| = n$. Suppose $x^t \in H$ then $\exists 0 \leq k < n, x^t = x^{nq+k}$ so $x^t = x^k \in H$.\\
    2 is trivially true.
\end{addmargin}
The proof above shows how to reduce $x^t$ into $x^k$ where $0 \leq k < |x|$.
\subsubsection*{Cyclic groups and gcd}
Given $x \in G, m, n \in \mathbb{Z}$. If $x^n = x^m = 1$, then denote $d = gcd(m, n), x^d = 1$. In particular $x^m = 1 \rightarrow |x||m$.\\
\textbf{Proof:}
\begin{addmargin}[1em]{0em}
    By the Euclidean Algorithm, $\exists r, s$ such that $mr+ns = d$, therefore $x^d = x^{mr+ns} = 1$.\\
    Then when $x^m = 1$, we have that $x^{|n|} = x^m = 1$, therefore $x^{(|x|, m)} = 1$, but since $|x|$ is the smallest such positive integer, $(|x|, m) = |x|$. Therefore $|x||m$.
\end{addmargin}
\subsubsection*{Cyclic groups isomorphic to $\mathbb{Z}/\mathbb{Z}n$}
\begin{enumerate}
    \item Given $n \in \mathbb{Z}^+, |\braket{x}| = |\braket{y}| = n$, then the map
    \begin{equation*}
        \begin{split}
            \phi: \braket{x} \rightarrow \braket{y}\\
            x^k \rightarrow y^k
        \end{split}
    \end{equation*}
    is well-defined and isomorphic.
    \item Given $\braket{x}$ is infinite, then the map
    \begin{equation*}
        \begin{split}
            \phi: \mathbb{Z} \rightarrow \braket{x}\\
            k \rightarrow x^k
        \end{split}
    \end{equation*}
    is well-defined and isomorphic
\end{enumerate}
\textbf{Proof:}
\begin{addmargin}[1em]{0em}
    For 1, first we NTS that the definition is well-defined:\\
    Given $x^r = x^s$, WTS $\phi(x^r) = \phi(x^s)$:\\
    Since $x^r = x^s \rightarrow x^{r-s} = 1 \rightarrow n|(r-s)$\\
    Therefore $x^s = x^{nk+r}$ for some $k$, and therefore $\phi(x^s) = \phi(x^{nk+r}) = \phi(x^{nk}\phi(x^r)) = \phi(x^r)$.\\
    Therefore the mapping is well-defined.\\
    Then it is easy to prove that $\phi$ is a homomorphism and since $|\braket{x}| = |\braket{y}|$, we only NTS $\phi$ is surjective, which is also obvious since $\forall y^k, \exists x^k, \phi(x^k) = y^k$.\\
    Therefore $\phi$ is an isomorphism.\\
    For 2, first we NTS that $\phi$ is injective:\\
    Given $r, s \in \mathbb{Z}$ such that $\phi(r) = \phi(s)$, we have that $x^r = \phi(r) = \phi(s) = x^s$.\\
    Therefore since $|\braket{x}| = \infty, x^r = x^s \rightarrow r = s$.\\
    Therefore $\phi$ is injective.\\
    It is obvious that $\phi$ is surjective.\\
    Therefore $\phi$ is isomorphic.
\end{addmargin}
We use $Z_n$ to denote the cyclic groups of order $n$.
\subsection*{Structure of cyclic groups}
We discuss which powers of $x$ generates $\braket{x}$:
\subsubsection*{Order of powers of $x$}
Let $G$ be a group, let $x \in G, a \in \mathbb{Z} - \{0\}$.
\begin{enumerate}
    \item If $|x| = \infty$, then $|x^a| = \infty$.
    \item If $|x| = n < \infty$, then $|x^a| = \frac{n}{(n, a)}$.
    \item In particular, if $|x| = n < \infty$ and $a$ is a positive integer dividing $n$, then $|x^a| = \frac{n}{a}$.
\end{enumerate}
\textbf{Proof:}
\begin{addmargin}[1em]{0em}
    For 1, suppose $|x^a| = k < \infty$, then $(x^a)^k = x^{ak} = 0 \rightarrow |x| < ak < \infty$. Contradiction.\\
    For 2, we denote $|x^a| = k, (n, a) = d, n = db, a = dc$. Then we only NTS that $|x^a| = k | \frac{n}{(n, a)} = b$ and $b | k$:\\
    Since $(x^a)^k = 0 \rightarrow n = db | ak = dck \rightarrow b | ck$.\\
    Since $(n, a) = d \rightarrow b, c$ coprime.\\
    Therefore $b | k$.\\
    We also know that $(x^a)^d = x^{ad} = x^{nc} = 0 \rightarrow k | b$.\\
    Therefore $b = k$.\\
    For 3, this is just a special case of 2 where $(n, a) = a$.
\end{addmargin}
\subsubsection*{Powers of $x$ that generate $\braket{x}$}
\begin{enumerate}
    \item Given $|x| = \infty, H = \braket{x^a} \iff a = \pm 1$.
    \item Given $|x| = n < \infty, H = \braket{x} \iff (a, n) = 1$.
\end{enumerate}
\textbf{Proof:}
\begin{addmargin}[1em]{0em}
    For 1, suppose $\braket{x^a} = \braket{x}$, then $\exists k \neq 0$ such that $(x^a)^k = x$.\\
    Therefore $x^{ak-1} = 0 \rightarrow a = \pm 1$.\\
    For 2, this is just using the above result.
\end{addmargin}
\subsubsection*{Structure of cyclic group}
Given $H = \braket{x}$ a cyclic group,
\begin{enumerate}
    \item Every subgroup of $H$ is cyclic. If $K \leq H$ then either $K = 1$ or $K = \braket{x^d}$ where $d$ is the smallest positive integer such that $x^d \in K$.
    \item If $|H| = \infty, \forall a \neq b \in \mathbb{Z}, \braket{x^a} \neq \braket{x^b}$. And $\forall m \in \mathbb{Z}, \braket{x^m} = \braket{x^{|m|}}$.
    \item If $|H| = n < \infty$, then for each positive integer $a$ that divides $n, \exists! K \leq H, |K|= a$. This subgroup $K$ is defined as $\braket{x^d}, d = \frac{n}{a}$. Furthermore for every integer $m, \braket{x^m} = \braket{x^{(m, n)}}$, i.e. the subgroups form a bijection with the positive divisors of $n$.
\end{enumerate}
\textbf{Proof:}
\begin{addmargin}[1em]{0em}
    For 1, suppose $K \neq 1$, then $\exists x^a \in K$.\\
    Therefore there must $\exists x^d$ where $d$ is the smallest such positive integer.\\
    We WTS that $K = \braket{x^d}$. Suppose $\exists x^a \in K, d \nmid a$, then $a = dq + r$ with $r < d, a^r \in K$, contradiction, therfore $\nexists$ such $x^a$.\\
    Therefore $K = \braket{x^d}$.\\
    For 2 this is obvious.\\
    For 3, first it is obvious that $|\braket{x^d}| = \frac{n}{(n, d)} = a$.\\
    Then we NTS that suppose $\exists \braket{x^b}$ such that $|\braket{x^b}| = a$, then $\braket{x^b} = \braket{x^d}$:\\
    Since $\braket{x} = \{1, x^b, x^{2b}, ..., x^{(a-1)b}\}$, we know that $x^{ab} = 1 \rightarrow n | ab$.\\
    Since $a | n, n | ab \rightarrow d | b$. Therfore $x^b \in \braket{x^d}$, i.e. $\braket{x^b} \subseteq \braket{x^d}$.\\
    Since $|\braket{x^b}| = |\braket{x^d}|$ and $\braket{x^b} \subseteq \braket{x^d} \rightarrow \braket{x^b} = \braket{x^d}$.\\
    Then the fact that $\braket{x^m} = \braket{x^{(n, m)}}$ naturally follows.
\end{addmargin}
\end{document}