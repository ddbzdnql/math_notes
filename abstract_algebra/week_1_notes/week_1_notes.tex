\documentclass{article}
\usepackage[utf8]{inputenc}
\usepackage{amsmath}
\usepackage{scrextend}
\usepackage{setspace}
\usepackage{amsfonts}
\usepackage{braket}

\title{Week 1 Notes\\
\large{Abstract Algebra}}
\author{shaozewxy }
\date{May 2022}

\doublespacing
\begin{document}

\maketitle

\setcounter{secnumdepth}{0}
\section{2022.05.05}
\subsection{Ch. 2 Dihedral Groups}
A dihedral group is a group whose elements are symmetries of regular $n$-gon. Let $D_{2n}$ denote the set of symmetries of a regular $n$-gon.\\
Each symmetry of the $n$-gon can be seen as a permutation $\sigma$ of $\{1,2,3,...,n\}$, where if the symmetry puts vertex $i$ in the place of vertex $j$, then $\sigma$ sends $i$ to $j$. The function of composition is therefore the composition of permutations.\\
This defines $D_{2n}$ as a group. The closure, associativity, identity and reverse can be proven easily.\\
We then show that:
\begin{equation*}
    |D_{2n}| = 2n
\end{equation*}
First, it is obvious that determining two vertices of the $n$-gon determines the $n$-gon as a whole.\\
We first look at vertex $1$, it can be sent to any $i \in \{1,2,3,...,n\}$.\\
Then we look at vertex $2$, because $2$ is immediately next to $1$, $2$ can either be sent to $i+1$ or $i-1$. Turns out, both are achievable through reflection.\\
Therefore, there are $n \cdot 2 = 2n$ ways of determining vertices $1$ and $2$, i.e. $|D_{2n}|=2n$.\\

\subsection{Generator and Relation}
Taking $D_{2n}$ as an example, let $r = \textrm{clockwise roration about the origin by } \pi / 2, s = \textrm{reflection about the line from the origin to } 1$. In general, $r,s$ can compose any element in $D_{2n}$:
\begin{addmargin}[1em]{0em}
    \textbf{(1)} $1,r,r^2,...,r^{n-1}$ are all distinct and $r^n=1$, i.e. $|r|=n$.\\
    For each of these permutations, the position of $1$ will be different, meaning these are all distinct permutations.\\
    \textbf{(2)} $|s|=2$.\\
    This is obvious.\\
    \textbf{(3)} $\forall i, s \neq r^i$.\\
    If $i \neq 0$, then $1$ will not be mapped to $1$ in $r^i$, but $1$ will be mapped to $1$ in $s$.\\
    If $i = 0$, then $2$ will be mapped to $n$ in $s$.\\
    Either case, $s \neq r^i$.\\
    \textbf{(4)} $\forall 0 \leq i,j \leq n-1, \textrm{ with }i\neq j, sr^i \neq sr^j$.\\
    Therefore,
    \begin{equation*}
        D_{2n} = \{1,r,r2,...,r^{n-1},s,sr,sr^2,...,sr^{n-1}\}
    \end{equation*}
    If $\exists \textrm{ such } i,j,$ then $|r| <= |j-i| \neq n$.\\
    \textbf{(5)} $rs=sr^{-1}$.\\
    As stated in the book, work out the positions of $1, 2$ after each of the permutations.\\
    \textbf{(6)} $\forall 0 \leq i \leq n, r^i s=sr^{-i}$.\\
    This can be proven by induction from \textbf{(5)}
\end{addmargin}
Given these facts, we see that elements in $D_{2n}$ can be written in terms of $r, s$, making calculations easier:
\begin{equation*}
    (sr^9)(sr^6)=s(r^9s)r^6=s(sr^{-9})r^6=s^2r^{-3}=r^{-3}
\end{equation*}
\section{2022.05.06}
\subsection{Generator}
We can abstract the role of $r, s$ in $D_{2n}$ to define a generator of a group:\\
A subset $s$ of elements of group $G$ such that $\forall x \in G$ can be written as a (finite) product of elements of $S$, is called a \textit{generator} of $G$.\\
We use the notion $G = \braket{S}$ to denote the fact.\\
Example:\\
$1 \in \mathbb{Z}$ is a generator for $(\mathbb{Z},+)$ the additive group of integers, i.e., $(\mathbb{Z}, +)=\braket{1}$.
\subsection{Relation}
Any equation in $G$ that the generators satisfy is called a \textit{relation}. In the case of $D_{2n}$, $r^n=1, s^2=1, rs=sr^{-1}$ are all relations. Additionally, these three relations can derive any other relations between elements in $D_{2n}$.
\subsection{Presentation}
If a group $G$ is generated by $S$, and $\exists$ a set of relations ${R_1, R_2, ..., R_m}$ such that any relations in $S$ can be derived from them, then we say the generators and relations are a \textit{presentation} of $G$:
\begin{equation*}
    G=\braket{S|R_1, R_2, ..., R_m}
\end{equation*}
For example:
\begin{equation*}
    D_{2n}=\braket{\{r,2\}|r^n=s^2=1, rs=sr^{-1}}
\end{equation*}
The subtleties with choosing a presentation of a group:\\
1. Cannot easily determine if two elements are equal.\\
$\braket{x_1, y_1 | x_1^2=y_1^2=(x_1y_1)^2=1}$ has order $4$, while\\
$\braket{x_2, y_2 | x_2^3=y_2^3=(x_2y_2)^3=1}$ has order $12$ \textbf{(exercise)}.\\
2. Some derived relations could cause the group to collapse.\\
$\braket{x,y|x^n=y^2=1, xy=yx^2}$ seems to similarly define $D_{2n}$ because using $xy=yx^2$ we can again write every element in the group as $y^ix^j$.\\
However, consider the hidden relation $x=x\cdot 1=xy^2$, we have $x=x^4$ \textbf{(exercise)}, collapsing this group to order at most $6$.\\
Another example:\\
$\braket{u,v|u^4=v^3=1, uv=v^2u^2}$ actually collapses to $\{1\}$ \textbf{(exercise)}.
\section{2022.05.12}
\subsection{Symmetric groups}
We use $\textbf{S}_n$ to denote the \textit{symmetric group of degree $n$}.\\
$|\textbf{S}_n| = n!$
\subsection{Cycle decomposition}
A cycle $(a_1 a_2 ... a_{m_1})$ will map $a_1 \rightarrow a_2, a_2 \rightarrow a_3, ..., a_{m_1 - 1} \rightarrow a_{m_1}, a_{m_1} \rightarrow a_1$\\
We claim that for each $\sigma \in \textbf{S}_n$ can be represented as $k$ disjoint cycles.
\subsection{Rings}
A ring $R$ is a set together with two binary operations $+,\times$, satisfying:\\
\textbf{i.} $(R,+)$ is an abelian group,\\
\textbf{ii.} $\times$ is associative,\\
\textbf{iii.} $(a+b)\times c = a\times c + b\times c, a\times(b+c) = a\times b + a\times c$.\\
A ring $R$ is said to have an \textit{identity} if there is an element $1 \in R$ such that
\begin{equation*}
    \forall a \in R, 1 \times a = a \times 1 = a
\end{equation*}
Having an identity forces $R$ to be commutative:
\begin{equation*}
    (1+1)(a+b) = 1(a+b) + 1(a+b) = a+b + a+b = (1+1)a + (1+1)b = a+a + b+b \Rightarrow a+b = b+a
\end{equation*}
\textit{Field} is a special type of the ring:
A ring $R$ with identity, where $1 \neq 0$, is called a \textit{division ring} if $\forall x \in R x \neq 0 \exists x^{-1} \in R \textrm{ s.t. } x \cdot x^{-1} = x^{-1} \cdot x = 1$.\\
A commutative division ring, is called a \textit{field}.\\
\subsection{Examples of Rings:}
1. $\mathbb{Z}$ is a commutative ring with identity. However, $(\mathbb{Z}-\{0\}, \times)$ is not a group.\\
2. $\mathbb{Q}, \mathbb{R}, \mathbb{C}$ are all commutative rings with identity, and they are all fields.\\
3. $\mathbb{Z}/n\mathbb{Z}$ the quotient group is a commutative ring with identity(the element $\Bar{1}$).\\
The first noncommutative ring is called the \textit{Hamilton Quaternion}:\\
4. $\mathbb{H}$ denotes a collection of elements of the form $a+bi+cj+dk$, with $a,b,c,d \in \mathbb{R}$, and the addition is defined by component-wise addition.\\
The multiplication is defined using distribution with:\\
$i^2=j^2=k^2=-1, ij=-ji=k, jk=-kj=i, ki=-ik=j$\\
For example, $(1+i+2j)(j+k) = -2+2i+2k$.\\
The identity is $1+0i+0j+0k$, and $\mathbb{H}$ over both $\mathbb{R}$ and $\mathbb{Q}$ are both division ring, with the inverse being:\\
\begin{equation*}
    (a+bi+cj+dk)^{-1} = \frac{a-bi-cj-dk}{a^2+b^2+c^2+d^2}
\end{equation*}
5. Rings of functions are defined with $(f+g)(x) = f(x) + g(x), (f \cdot g)(x) = f(x) \cdot g(x)$.\\
6. $2\mathbb{Z}$ the ring of even integers don't have an identity.\\
\subsection{Properties of Rings:}
1. $\forall a \in R, 0a = a0 = 0$\\
\textbf{Proof:}
\begin{equation*}
    a0 = a(0+0) = a0 + a0 \Rightarrow a0 = 0
\end{equation*}
2. $(-a)b = a(-b) = -ab$\\
\textbf{Proof:}
\begin{equation*}
    (-a)b + ab = (-a+a)b = 0 = -ab + ab \Rightarrow (-a)b = -ab
\end{equation*}
3. $(-a)(-b) = ab$\\
\textbf{Proof:}
\begin{equation*}
    (-a)(-b) + (-a)b = -a(-b+b) = 0 = (-a)b + ab \Rightarrow (-a)(-b) = ab
\end{equation*}
4. If $R$ has identity $1$, then then identity is unique and $-a = (-1)a$\\
\textbf{Proof:}
\begin{equation*}
    (-1)a + a = (-1)a + 1a = (-1+1)a = 0 = -a + a \Rightarrow -a = (-1)a
\end{equation*}
\subsubsection{Structures within rings}
\textbf{Zero divisor}\\
A nonzero element $a \in R$ is a \textit{zero divisor} if $\exists b \in R b \neq 0$ such that $ab=0$ or $ba=0$.\\
\textbf{Unit}\\
If $R$ has an identity $1\neq 0$. An element $u \in R$ is a \textit{unit} if $\exists v \in R$ such that $uv = vu = 1$.\\
The set of units in $R$ is denoted $R^\times$.\\\\
It is clear that $R^\times$ is a group under multiplication, referred to as \textit{group of units of $R$}.\\
A field can be seen as a ring with an identity $1 \neq 0$ where every non-zero element is a unit.
\subsubsection{Zero divisor cannot be a unit:}
$a$ is a unit, therefore $\exists b \in R, ab=0$.\\
If also $\exists v \in R, av = 1$, then $vab = b \neq 0$, but $v\cdot 0 = 0$, contradiction.
\subsubsection{Examples:}
1. $\mathbb{Z}$ the ring of integers has no zero divisors, and the units are $\mathbb{Z}^\times = \pm 1$.\\
2. $\mathbb{Z}/n\mathbb{Z}$ with $n \geq 2$, The units are $\{\Bar{u}|u\textrm{ coprime with } n\}$.\\
Therefore, $(\mathbb{Z}/n\mathbb{Z})^\times$ denotes all these coprime elements.\\
If $a$ is not coprime with $n$, then we show $a$ is a zero divisor in $\mathbb{Z}/n\mathbb{Z}$:\\
Let $d = \textrm{gcd}(a, n)$, then $n | a \cdot \frac{n}{d}$.\\
3. Let $D \in \mathbb{Q}$ while not being a perfect square. Denote
\begin{equation*}
    \mathbb{Q}(\sqrt{D}) = \{a+b\sqrt{D}|a,b \in \mathbb{Q}\}
\end{equation*}
This is closed under addition and multiplication, and therefore it is a sub ring of $\mathbb{C}$\\
Now if $a,b$ not both $0$, then $a^2 != Db^2$, and therefore $\forall a+b\sqrt{D}$, we have $(a+b\sqrt{D})\cdot \frac{a-b\sqrt{D}}{a^2-Db^2}$, i.e. every element also has an inverse.\\
This shows that every element is either a unit or $0$, therefore, $\mathbb{Q}(\sqrt{D})$ is a field, called a \textit{quadratic field}.
\subsubsection{Integral domain}
A commutative ring with $1 \neq 0$ is called an \textit{integral domain} if it has no zero divisors.\\
\subsection{Cancellations in integral domains}
If $a,b,c \in R$ and not zero divisors. If $ab=ac$, then either $a=0$ or $b=c$. This holds true for all elements in an integral domain.\\
\textbf{Proof:}\\
\begin{addmargin}[1em]{0em}
    $ab=ac \Rightarrow a(b-c) = 0 \Rightarrow a=0 \textrm{or} (b-c) = 0$
\end{addmargin}
\subsubsection{Finite integral domain is a field}
\textbf{Proof:}\\
\begin{addmargin}[1em]{0em}
    We only need to show that every element is a unit:\\
    Given $a \in R, a\neq 0$, we map $R \rightarrow aR$.\\
    because cancellation principle, every distinct element in $R$ will provide a distinct result, i.e. $|aR| = |R|$.\\
    Then according to the pigeonhole principle, there must be some $b \in R$ such that $ab = 1$, making $R$ a field.
\end{addmargin}
\subsubsection{Case study: Quadratic Integer Rings}
Given $D$ a squarefree integer. Define $\mathbb{Z}[\sqrt{D}] = \{a+b\sqrt{D}|a,b \in \mathbb{Z}\}$.\\
It is clear that $\mathbb{Z}[\sqrt{D}]$ is a subring of $\mathbb{Q}[\sqrt{D}]$.\\
If $D \equiv 1\ mod\ 4$ then the subset
\begin{equation*}
    \mathbb{Z}[\frac{1+\sqrt{D}}{2}] = \{a+b\frac{1+\sqrt{D}}{2}|a,b \in \mathbb{Z}\}
\end{equation*}
is also a subring. The multiplicate closure holds because $D \equiv 1\ mod\ 4$, and $a+b\frac{1+\sqrt{D}}{2} \cdot c+d\frac{1+\sqrt{D}}{2} = ac+bd \cdot \frac{1+D+2\sqrt{D}}{4} + (ad+bc)\frac{\sqrt{D}+1}{2} = (ac+bd\cdot \frac{D-1}{4}) + (ad+bc+bd)\frac{1+\sqrt{D}}{2}$.\\
Then we can define
\begin{equation*}
    O = O_{\mathbb{Q}[\sqrt{D}]} = \mathbb{Z}[\omega]=\{a+b\omega|a,b\in \mathbb{Z}\}
\end{equation*}
where
\begin{equation*}
    \omega = \begin{cases}
    \sqrt{D},& \textrm{if } D \equiv 2,3\ mod\ 4\\
    \frac{1+\sqrt{D}}{2}, & \textrm{if } D \equiv 1\ mod\ 4
    \end{cases}
\end{equation*}
This is called the \textit{ring of integers} in the qudratic field $\mathbb{Q}[\sqrt{D}]$.\\
When we put $D=-1$, this is the complex ring $\{a+bi|a,b \in \mathbb{Z}\}$.\\
Define \textit{field norm} $N:\mathbb{Q}(\sqrt{D})\rightarrow \mathbb{Q}$ by
\begin{equation*}
    N(a+b\sqrt{D}) = (a+b\sqrt{D})(a-b\sqrt{D}) = a^2 - Db^2 \in \mathbb{Q}
\end{equation*}
This can be used as a measure of the size of an element in the field.\\
Clearly $N$ id multiplicative, in that $N(\alpha\beta) = N(\alpha)N(\beta)$.\\
On the subring $O$, we can see that
\begin{equation*}
    N(a+b\omega) = (a+b\omega)(a+b\Bar{\omega}) = \begin{cases}
    a^2 - Db^2,& D\equiv 2,3\ mod\ 4\\
    a^2+ab+\frac{1-D}{4}b^2
    \end{cases}
\end{equation*}
\subsubsection{$\alpha$ is a unit in $O$ iff $N(\alpha)=\pm 1$}
\textbf{Proof:}
If $\alpha \in O$ has $N(\alpha) = \pm 1$, then $a \cdot \pm\Bar{a} = 1$, therefore $\alpha$ is a unit.\\
If $\alpha$ is a unit, i.e., $\alpha\beta = 1$, then $N(\alpha)N(\beta) = 1$, and because norm is integer, $N(\alpha) = \pm1$.
\end{document}