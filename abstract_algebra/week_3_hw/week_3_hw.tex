\documentclass{article}
\usepackage[utf8]{inputenc}
\usepackage{amsmath}
\usepackage{scrextend}
\usepackage{setspace}
\usepackage{amsfonts}
\usepackage{braket}
\usepackage{amssymb}

\title{Abstract Algebra\\
\large{Week 3 HW}}
\author{shaozewxy }
\date{August 2022}

\doublespacing
\begin{document}

\maketitle

\setcounter{secnumdepth}{0}
\section*{1}
(1.7.11)\\
Write out the cycle decomposition of the eight permutation in $S_4$ corresponding to the elements of $D_8$ given by the action of $D_8$ on the vertices of a square.
\begin{itemize}
    \item[$1$:] $1$
    \item[$r$:] $(1\ 2\ 3\ 4)$
    \item[$r^2$:] $(1\ 3)(2\ 4)$
    \item[$r^3$:] $(1\ 4\ 3\ 2)$
    \item[$s$:] $(2\ 4)$
    \item[$sr$:] $(1\ 2)(3\ 4)$
    \item[$sr^2$:] $(1\ 3)$
    \item[$sr^3$:] $(1\ 4)(2\ 3)$
\end{itemize}
\section*{2}
(1.7.17)\\
Let $G$ be a group and let $G$ act on itself by left conjugation, so each $g \in G$ map $G$ to $G$ by
\begin{equation*}
    x \mapsto gxg^{-1}
\end{equation*}
For fixed $g \in G$, prove that conjugation by $g$ is an isomorphism from $G$ onto itself. Deduce that $x$ and $gxg^{-1}$ have the same order for all $x$ in $G$ and that for any subset $A$ of $G, |A| = |gAg^{-1}|$.
\begin{addmargin}[1em]{0em}
    Fixing $g \in G$, define $\phi_g: G \rightarrow G$ by
    \begin{equation*}
        \phi_g(x) = gxg^{-1}
    \end{equation*}
    First NTS $\phi_g$ is injective:\\
    $\forall x, y \in G, \phi_g(x) = \phi_g(y) \rightarrow gxg^{-1} = gyg^{-1} \rightarrow g^{-1}(gxg^{-1})g = g^{-1}(gyg^{-1})g$\\
    Therefore $x = y$, i.e. $\phi_g$ is injective.\\
    Then NTS $\phi_g$ is surjective:\\
    $\forall x \in G, \exists g^{-1}xg \in G, \phi_g(g^{-1}xg) = g(g^{-1}xg)g^{-1} = g$.\\
    Therefore $\phi_g$ is surjective.\\
    Since $\phi_g$ is defined from an action, it is already a homomorphism, and therefore $\phi_g$ is an isomorphism.\\\\
    Now since $\phi_g$ is an isomorphism, $\exists \phi_g^{-1}, \forall x \in G, \phi_g^{-1}(\phi_g(x)) = 1$.\\
    Given $a, b \in G, \phi_g(a) = b$, and assume $|a| = m, |b| = n$.\\
    This means $\phi_g(a)^m = \phi_g(a^m) = 1 = b^m$\\
    Therefore $|b| \leq |a|$.\\
    Similarly $\phi_g^{-1}(b)^n = \phi_g^{-1}(b^n) = 1 = a^n$.\\
    Therefore $|a| \leq |b|$.\\
    This shows that $\forall a, b \in G, \phi_g(a) = b,$we have $|a| = |b|$.\\
    Therefore we have $\forall x \in G, |x| = |gxg^{-1}|$.\\\\
    We can similarly define a map $\psi_g: A \mapsto gAg^{-1}$ by
    \begin{equation*}
        \forall x \in A, \psi_g(x) = gxg^{-1}
    \end{equation*}
    First NTS $\psi_g$ is well defined:\\
    $\forall x \in A, \psi_g(x) = gxg^{-1}$ which by definition is in $gAg^{-1}$. Therefore $\psi_g$ is well-defined.\\
    Then NTS $\psi_g$ is both injective and surjective. Which is similar to what we did for showing $\phi_g$ is invertible. Therefore $\exists$ a bijection $\psi_g$ between $A$ and $gAg^{-1}$.\\
    Therefore $|A| = |gAg^{-1}|$
\end{addmargin}
\section*{3}
(1.7.19)\\
Let $H$ be a subgroup of the finite group $G$ and let $H$ act on $G$ by left multiplication. Let $x \in G$ and let $\mathcal{O}$ be the orbit of $x$ under the action of $H$. Prove that the map
\begin{equation*}
    H \rightarrow \mathcal{O} \textrm{  defined by } h \mapsto hx
\end{equation*}
is a bijection. From this and the preceeding exercise deduce \textbf{Lagrange's Theorem}:\\
If $G$ is a finite group and $H$ is a subgroup of $G$ then $|H|$ divides $|G|$.
\begin{addmargin}[1em]{0em}
    This map is by nature surjective.\\
    Only NTS it is injective:\\
    $\forall g, h \in H, gx = hx \rightarrow (gx)x^{-1} = (hx)x^{-1} \rightarrow g = h$.\\
    Therefore the map is injective and therefore a bijection.
\end{addmargin}
\section*{4}
(2.2.7)\\
Let $n \in \mathbb{Z}$ with $n \geq 3$. Prove the following:
\begin{itemize}
    \item[(a)] $Z(D_{2n}) = 1$ if $n$ is odd
    \item[(b)] $Z(D_{2n}) = \{1, r^k\}$ if $n = 2k$.
\end{itemize}
\begin{addmargin}[1em]{0em}
    Given $s \in D_{2n}$, we have
    \begin{equation*}
        \begin{split}
            \forall sr^i \in D_{2n}, sr^is(sr^i)^{-1} & = sr^isr^{-i}s \\
            & = sr^{2i}
        \end{split}
    \end{equation*}
    $sr^{2i} = s \rightarrow r^{2i} = 1 \rightarrow 2i = n$ or $i = 0$ since $n$ is odd, it can only be that $i=0$, i.e. $sr^0 = s.$\\
    However $sr^is^{-1} = r^{-i}$. Therefore $\nexists sr^i \in Z(D_{2n})$\\
    Similarly,
    \begin{equation*}
            \forall r^i \in D_{2n}, r^isr^{-i} = r^{2i}s
    \end{equation*}
    $r^{2i}s = s \rightarrow r^{2i} = 1$ and we can see that $r^{2i}$ must be $1$.\\
    This shows that only $1$ is in $Z(D_{2n})$ when $n$ is odd.\\
    Using the same logic from above, we can see that when $n = 2k$, we can let $i = k$ so that $r^i = r^k \in Z(G)$.\\
    Therefore $\{1, r^i\} = Z(G)$ when $n = 2k$.
\end{addmargin}
\section*{5}
(2.2.10)\\
Let $H$ be a subgroup of $G$ and any nonempty subset $A$ of $G$ define $N_H(A)$ to be the set $\{h \in H | hAh^{-1} = A\}$. Show that $N_H(A) = N_G(A) \cap H$ and deduce that $N_H(A)$ is a subgroup of $H$.
\begin{addmargin}[1em]{0em}
    First NTS $N_H(A) \subseteq N_G(A) \cap H$:\\
    $\forall h \in N_H(A)$, since $h \in H \leq G$ and $hAh^{-1} = A \rightarrow h \in N_G(A)$.\\
    Therefore $h \in N_G(A) \cap H$, i.e. $N_H(A) \subseteq N_G(A) \cap H$.\\
    Then NTS $N_G(A) \cap H \subseteq N_H(A)$:\\
    $\forall h \in N_G(A) \cap H$, since $h \in H$ and $hAh^{-1} = A$, $h \in N_H(A)$.\\
    Therefore $N_G(A) \cap H \subseteq N_H(A)$.\\
    Therefore $N_G(A) \cap H = N_H(A)$.\\
    Since $H \leq G, N_G(A) \leq G, \rightarrow N_H(A)=N_G(A) \cap H \leq G$.\\
    Since $N_H(A) \subseteq H, \rightarrow N_H(A) \leq H$.
\end{addmargin}
\section*{6}
(2.3.17)\\
Find a presentation for $Z_n$ with one generator.
\begin{addmargin}[1em]{0em}
    $Z_n = \{z|z^n=1\}$
\end{addmargin}
\section*{7}
(2.3.25)\\
Let $G$ be a cyclic group of order $n$ and let $k$ be an integer relatively prime to $n$. Prove that the map $x \mapsto x^k$ is surjective. Use Lagrange's Theorem to prove the same is true for any finite group of order $n$.
\begin{addmargin}[1em]{0em}
    Since $k$ relatively prime to $n$, this means $gcd(k, n) = 1$.\\
    Therefore $\exists r, s$ such that $rk+sn = 1 \rightarrow \forall a \in \mathbb{Z}, ark+asn = a$.\\
    Therefore $x^{ark+asn} = x^a = x^{ark} \cdot 1$, i.e. $(x^{ar})^k = x^a$.\\
    For any finite group of order $n$:\\
    Given $x \in G$, by Lagrange's Theorem we know $|x||k$ and thus $k$ also relatively prime to $|x|$.\\
    Therefore this is also true for any finite group of order $n$.
\end{addmargin}
\section*{8}
(3.1.9)
\section*{9}
(3.1.32)
\section*{10}
(3.1.33)
\end{document}