\documentclass{article}
\usepackage[utf8]{inputenc}
\usepackage{amsmath}
\usepackage{scrextend}
\usepackage{setspace}
\usepackage{amsfonts}
\usepackage{braket}

\title{HW 3\\
\large{Abstract Algebra}}
\author{shaozewxy }
\date{May 2022}

\doublespacing
\begin{document}
\maketitle
\setcounter{secnumdepth}{0}
\section{1.7.11}
$s = (1\ 2) (3\ 4), r = (1\ 2\ 3\ 4)$, then we have\\
$r^2 = (1\ 3)(2\ 4), r^3 = (1\ 4\ 3\ 2)$, and\\
$sr = (1\ 3), sr^2 = (1\ 4)(2\ 3), sr^3 = (2\ 4)$
\section{1.7.17}
Given $g \in G$, define $\phi_g: G \rightarrow G$ as
\begin{equation*}
    \phi_g(x) = gxg^{-1}
\end{equation*}
First NTS $\phi_g$ a homomorphism:\\
$\phi_g(1) = g\cdot 1\cdot g^{-1} = g \cdot g^{-1} = 1$\\
$\forall x, y \in G, \phi_g(x) \cdot \phi_g(y) = gxg^{-1} \cdot gyg^{-1} = gxyg^{-1} = \phi_g(xy)$\\
Therefore $\phi_g$ a homomorphism.\\
Then NTS $\phi_g$ is surjective:\\
$\forall x \in G, \exists g^{-1}xg \in G, \phi_g(g^{-1}xg) = g(g^{-1}xg)g^{-1} = x$, therefore $\phi_g$ is surjective.\\
Because $\phi_g$ maps $G$ to $G$ and surjective, $\phi_g$ is bijective, i.e. $\phi_g$ an isomorphism.\\
Then NTS $|x| = |gxg^{-1}|$:\\
Suppose $|x| = n$, then $\phi_g(x^n) = \phi_g(x)^n = 1$, i.e. $(gxg^{-1})^n = 1 \rightarrow |gxg^{-1}| <= n$.\\
If $|gxg^{-1}| = m < n$, then $\phi_g^{-1}(gxg^{-1}^m) = \phi_g^{-1}(gxg^{-1})^m = x^m = 1$ this is contradictory to $|x| = n$, therefore $|gxg^{-1}| = n = |x|$.\\
Then NTS $\forall A \subseteq G, |A| = |gAg^{-1}|$:\\
There is a bijection $\psi_g: A \rightarrow gAg^{-1}$ defined as:
\begin{equation*}
    \psi_g(x) = \phi_g(x)
\end{equation*}
This is obviously an isomporhism and therfore $|A| = |gAg^{-1}|$
\section{1.7.19}
Given $x \in G$, we try to prove $\phi: H \rightarrow \mathcal{O}$ defined by
\begin{equation*}
    \phi(h) = hx
\end{equation*}
is a bijection:\\
Given $h_1, h_2 \in H$, if $\phi(h_1) = \phi(h2) \rightarrow h_1x = h_2x \rightarrow h_1 = h_2$, therefore $\phi$ is injective.\\
Given $x' \in \mathcal{O}, \exists h \in H$ such that $hx = x'$, therefore $\phi(h) = hx = x'$, therefore $\phi$ is surjective.\\
Therefore $\phi$ is a bijection and $|H| = \mathcal{O}$.\\
Then we NTS if $|G| < \infty$, then $|H|$ divided $|G|$. Because $\sim$ defined as
\begin{equation*}
    a \sim b \iff \exists h \in H, a = hb 
\end{equation*}
partitions $G$, then we have that each equivalence class of $\sim$ has order $|H|$, therefore $|H|$ divides $|G|$.
\section{2.2.7}
Suppose $sr^i \in Z(D_{2n})$, then we have $\forall j, sr^isr^j = ssr^{j-i} = sr^jsr^i = ssr^{i-j} \rightarrow i = j$. therefore it is impossible, i.e. $x \in Z(D_{2n}) \rightarrow x = r^i$.\\
Suppose $r^i \in Z(D_{2n})$, then $\forall sr^j \in D_{2n}, r^isr^j = sr^{j-i} = sr^jr^i = sr^{j+i} \rightarrow j \in \mathbb{Z}/2k\mathbb{Z}, 2j = 0$, therefore either $j = 0 \rightarrow x = r^0 = 1$ or $j = k \rightarrow x = r^k$.
\section{2.2.10}
Because $|H| = 2$, then say $H = \{1, a\}$, and we have $a^2 = 1$.\\
Suppose $x \in N_G(H)$, then $xax^{-1} =$ either $1$ or $a$.\\
If $xax^{-1} = 1$, then $xa = a \rightarrow a = 1$, contradiction. Therefore $xax^{-1} = a$, i.e $x \in C_G(H)$. Therefore $N_G(H) \subseteq C_G(H)$.\\
Because obviously $C_G(H) \subseteq N_G(H)$, we have $C_G(H) = N_G(H)$.
\section{2.3.17}
$\mathbb{Z}_n$ is generated by $\braket{1}$.\\
\section{2.3.25}
Because the map $\phi(x) = x^k$ maps $G$ to $G$, then in order to prove $\phi$ is surjective, we only NTS it is injective.\\
Because $k$ relatively prime to $n$, we know that $|\braket{x^k}| = \frac{n}{(n, k)} = n$, i.e. $(x^k)^1 \neq (x^k)^2 \neq ... \neq (x^k)^{n-1}$.\\
Then $\forall x^a \neq x^b \in G$, we have $\phi(x^a) = (x^a)^k = (x^k)^a \neq (x^k)^b = \phi(x^b)$, therefore $\phi$ is injective and therefore surjective.\\
Similarly we can see that $\exists a, b$ such that $ak+bn = 1$, then $\forall x^t \in G$, we have $x^{t(ak+bn)} = x^t \rightarrow (x^{at})^k \cdot (x^{tbn}) = (x^{at})^k = x^t$m therefore $\phi(x^{at}) = x^t$, therefore $\phi$ is surjective.\\
Then given $x \in G$, it is clear $x \in \braket{x}$ so we only NTS $|\braket{x}|$ also coprime with $k$. Due to LaGrange's Theorem, we know that $\braket{x}|$ divides $n$, then $|\braket{x}|$ must also be coprime with $k$.
\section{3.1.9}
$\phi(1+0i) = 1 + 0^2 = 1$.\\
$\phi(a_1+b_1i) \cdot \phi(a_2+b_2i) =(a_1^2 + b_1^2)(a_2^2 + b_2^2) = (a_1a_2)^2 + (a_1b_2)^2 + (b_1a_2)^2 + (b_1b_2)^2$
$= (a_1a_2)^2 + (b_1b_2)^2 - 2a_1a_2b_1b_2 + (a_1b_2)^2 + (b_1a_2)^2 + 2a_1a_2b_1b_2 = (a_1a_2 - b_1b_2)^ + (a_1b_2 + a_2b_1)^2 = \phi((a_1+b_1i)(a_2+b_2i))$
Therefore $\phi$ is a homomorphism.\\
$ker\ \phi = \{a+bi|a^2+b^2 = 1\}$
The fibers of $\phi$ are concentric circles around the origin of the plane.
\end{document}