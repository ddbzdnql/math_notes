\documentclass{article}
\usepackage[utf8]{inputenc}
\usepackage{amsmath}
\usepackage{scrextend}
\usepackage{setspace}
\usepackage{amsfonts}
\usepackage{braket}

\title{Week 5 Notes\\
\large{Abstract Algebra}}
\author{shaozewxy }
\date{May 2022}

\doublespacing
\begin{document}
\maketitle
\setcounter{secnumdepth}{0}
\section{4.1 Group Actions and Permutations}
\subsection{Basis Concepts}
$G$ a group acting on $A$ a nonempty set.\\
Then the map $\sigma_g:A \rightarrow A$ defined by
\begin{equation*}
    \sigma_g(a) = g\cdot a
\end{equation*}
is a permutation of $A$.\\
There is also a homomorphism $\phi:G \rightarrow S_A$ defined by
\begin{equation*}
    \phi(g) = \sigma_g
\end{equation*}
This homomorphism is called the \textbf{permutation representation} associated to this action.\\
The \textbf{kernel} of the action is defined as
\begin{equation*}
    \{g \in G|\forall a \in A, g\cdot a = a\}
\end{equation*}
The \textbf{stabilizer} of $a$ in $G$ is defined as
\begin{equation*}
    \{g \in G|g\cdot a = a\}
\end{equation*}
An action of $G$ on $A$ is \textbf{faithful} if its kernel is the identity.\\
From this we can deduce some results:
\begin{itemize}
\item The kernel of an action is the same as the kernel of the permutation representation associated to the action.
\item The kernel of the action is a normal subgroup of $G$.
\item $a, b \in G$ induce the same permutation $\iff a, b$ are in the same fiber of the kernel.
\item An action of $G$ on $A$ is also a faithful action of $G/ker\ \phi$ on $A$.
\item Fixing an element $a \in A$, the kernel of the action is contained in the stabilizer $G_a$.
\item In particular $ker\ \phi = \cap_{a \in A} G_a$.
\end{itemize}
\subsection{Group Action Defines an Equivalent Relation}
We define a relation on $A$ by
\begin{equation*}
    a \sim b \iff \exists g \in G, g\cdot b = a
\end{equation*}
For each $a \in A$, the number of elements in the equivalence class of $a$ is $|G:G_a|$, the index of $G_a$.\\
\textbf{Proof:}
\begin{addmargin}[1em]{0em}
First NTS this is an equivalent relation:\\
$1\cdot a = a \rightarrow a \sim a$\\
$a \sim b \rightarrow \exists g \in G, g\cdot b = a \rightarrow g^{-1}\cdot g\cdot b = g^{-1}\cdot a = (g^{-1}g)\cdot b = b \rightarrow b \sim a$\\
$a \sim b, b \sim c \rightarrow \exists g, h \in G, g \cdot b = a, h \cdot c = b \rightarrow gh \cdot c = g\cdot h \cdot c = g\cdot b = a \rightarrow a \sim c$\\
We denote $\mathcal{C}_a$ as
\begin{equation*}
    \mathcal{C}_a = \{g\cdot a| g \in G\}
\end{equation*}
Then we show that there is a bijection $\phi:\mathcal{C}_a \rightarrow G/G_a$ defined by
\begin{equation*}
    \phi(g \cdot a) = gG_a
\end{equation*}
Clearly $\phi$ is surjective since $\forall gG_a, \exists g\cdot a \in \mathcal{C}_a, \phi(g\cdot a) = gG_a$.\\
Then NTS $\phi$ is injective.\\
Given $g_1\cdot a \neq g_2\cdot a$, then suppose $g_1G_a = g_2G_a$.\\
Therefore $\exists g \in G_a, g_1g = g_2 \rightarrow g_2\cdot a = g_1g \cdot a = g_1\cdot a$. Contradictory to $g_1\cdot a \neq g_2\cdot a$. Therefore $g_1G_a \neq g_2G_a$, i.e $\phi$ injective.\\
Therefore $\phi$ is a bijection and therefore $|\mathcal{C}_a| = |G/G_a| = |G:G_a|$
\end{addmargin}
\textbf{Proof of Unique Cycle Decomposition:}\\
Given $\sigma \in S_n$ we create $G = \braket{\sigma}$ and let it act on $A = \{1, 2, ..., n\}$ by
\begin{equation*}
    \sigma^i \cdot x = \sigma^i(x)
\end{equation*}
Then given $x \in A$, we denote its orbit in $G, \mathcal{O} = \{\sigma^i\cdot(x)| \sigma^i \in G\}$.\\
From proof of the equivalence relation above we know that therre is a bijection $\phi: \mathcal{O} \rightarrow G/G_x$ defined by
\begin{equation*}
    \phi(\sigma^i \cdot x) = \sigma^iG_x
\end{equation*}
Because $G$ is cyclic, we know $G/G_x$ is also cyclic and has order $d$ the smallest positive integer such that $\sigma^d \in G_x$. Thererfore the elements of $G/G_x$ are
\begin{equation*}
    G_x, \sigma G_x, \sigma^2G_x, ..., \sigma^{d-1}G_x
\end{equation*}
Correspondingly the elements of $\mathcal{O}$ are
\begin{equation*}
    x, \sigma(x), \sigma^2(x), ..., \sigma^{d-1}(x)
\end{equation*}
Therefore for each element $x \in A$, $x$ is in a cycle for $\sigma \in S_n$.
\end{document}