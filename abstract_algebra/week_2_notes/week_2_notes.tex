\documentclass{article}
\usepackage[utf8]{inputenc}
\usepackage{amsmath}
\usepackage{scrextend}
\usepackage{setspace}
\usepackage{amsfonts}
\usepackage{braket}
\usepackage{amssymb}

\title{Abstract Algebra\\
\large{Week 2 Notes}}
\author{shaozewxy }
\date{August 2022}

\doublespacing
\begin{document}

\maketitle

\setcounter{secnumdepth}{0}
\section*{1.7 Group Actions}
The concept of an "action" is important for studying algebraic objects by seeing how it can act on other structures.
\subsection*{Definition of group action}
A \textbf{group action} of a group $G$ on a set $A$ is a map from $G \times A \rightarrow A$. Given $g \in G, a, b \in A$, the action is written as $g\cdot a \rightarrow b$. The action needs to satisfy the following properties:
\begin{enumerate}
    \item $\forall g_1, g_2 \in G, a\in A, g_1\cdot(g_2\cdot a) = g_1g_2\cdot a$.
    \item $\forall a \in A, 1\cdot a = a$.
\end{enumerate}
\subsection*{Group action defines permutation and homomorphism}
Given a group $G$ acting on $A$. Then $\forall g \in G, \exists \sigma_g: A \rightarrow A$ defined by
\begin{equation*}
    \sigma_g(a) = g\cdot a
\end{equation*}
We observe the following facts:
\begin{itemize}
    \item[(i)] For each $g\in G, \sigma_g$ is a permutation.
    \item[(ii)] The map from $G \rightarrow S_A$ defined by $g \rightarrow \sigma_g$ is a homomorphism.
\end{itemize}
\textbf{Proof:}
\begin{addmargin}[1em]{0em}
    For i, since $\sigma_g$ maps $A$ to itself, then only NTS $\sigma_g$ is injective to show $\sigma_g$ is bijective and therefore a permutation.\\
    Suppose $\exists a, b \in A, \sigma_g(a) = \sigma_a(b)$.\\
    Then $\sigma_g(a) = g\cdot a = \sigma_g(b) = g\cdot b \rightarrow g^{-1}\cdot (g\cdot a) - g^{-1}\cdot (g\cdot b) \rightarrow 1\cdot a = a = 1\cdot b = b$.\\
    Therefore $\sigma_g$ is injective and thus a permutation.\\
    For ii, define $\phi:G \rightarrow S_A$ by
    \begin{equation*}
        \phi(g) = \sigma_g
    \end{equation*}
    then $\forall g, h \in G, \phi(gh) = \sigma_{gh}, \phi(g)\phi(h) = \sigma_g\sigma_h$.\\
    $\forall x \in A, \sigma_{gh}(x) = gh\cdot x = g\cdot h\cdot x = \sigma_g\sigma_h(x)$.\\
    Therefore $\forall g, h \in G, \phi(gh) = \phi(g)\phi(h)$.\\
    The remaining criteria can be easily proved.
\end{addmargin}
The homomorphism above is called the \textbf{permutation representation} associated to the given action.
\subsection*{Examples of group actions}
\begin{enumerate}
    \item Define a group action by $\forall g\in G, a\in A, ga = a$.\\
    This is called a \textbf{trivial} action. The associated permutation representation mapping is also trivial as it maps all elements to the identity.\\
    A group action is said to be \textbf{faithful} if distinct elements induce distinct permutations of $A$, i.e. the associated permutation representation is injective.\\
    The \textbf{kernel} of an action is defined to be $\{g\in G | \forall a \in A, ga = a\}$, i.e. the elements that maps to identity in the associated permutation representaion.
    \item The scalar multiplication of a field $\mathbf{F}$ and a vector space $V$ is also an example of group actions with field $\mathbf{F}$ acting on the vector space $\mathbf{F}^n$ defined by
    \begin{equation*}
        \forall a\in \mathbf{F}, v = (v_1, ..., v_n) \in \mathbf{F}^n, av = (av_1, ..., av_n)
    \end{equation*}
    \item For any non-empty set $A$ the symmetric group $S_A$ acts on $A$ by $\sigma a = \sigma(a)$. The associated permutation association is just the identity map.
    \item Each element $\alpha \in D_{2n}$ defined a permuation $\sigma_{\alpha} \in S_n$ by fixing a labeling of the vertices.\\
    Therefore $D_{2n}$ defines a group action on $\{1, 2, ..., n\}$ by
    \begin{equation*}
        \forall \alpha \in D_{2n}, i \in \{1, 2, ..., n\}, \alpha \cdot i = \sigma_\alpha(i)
    \end{equation*}
    \item A finite group of order $n$ is isomorphic to some subgroup of $S_n$.\\
    We define a group action of $G$ acting on itself by
    \begin{equation*}
        \forall g, a \in G, g\cdot a = ga
    \end{equation*}
    Now we NTS the associated permutation representaion $\pi:G \rightarrow S_n$ is injective:\\
    Suppose $\pi(g) = \pi(h)$, then $\pi(g)g^{-1} = \pi(h)g^{-1} \rightarrow hg^{-1} = 1 \rightarrow h = g$.\\
    Therefore $\pi$ is injective and therefore there is a subgroup in $S_n$ that is isomorphic to $G$.
\end{enumerate}
\section*{2.5 Lattice of Subgroups of a Group}
The lattice displays the structure of a group. It is constructed as follow:
\begin{itemize}
    \item The subgroup $1$ is placed at the bottom.
    \item The group $G$ itself is placed at the top.
    \item For subgroups $A, B$ there will be a line connecting them if $A \le B$ and there is no proper subgroup between them.
\end{itemize}
\subsection*{Drawbacks and Properties of Lattice}
For infinite groups, we cannot draw a complete lattice.\\
Even for some finite groups, the lattice can be very complicated. For example, groups of order $2^n$.\\
However, for these cases, we can draw just part of the lattice which can still be very helpful.
\section*{2.4 Subgroups Generated by Subsets of a Group}
The concept of given a group $G$ and a subgroup $A \leq G$, is there a unique minimal subgroup that contains $A$, is a recurring theme.\\
In vector space this is called a span, and in group this is called \textbf{the subgroup generated by $A$}.
\subsection*{Proposition 8}
If $\mathcal{A}$ is a non-empty collection of subgroups of $G$, then the intersection of all members of $\mathcal{A}$ is also a subgroup $G$.\\
\textbf{Proof:}
\begin{addmargin}[1em]{0em}
    Denote
    \begin{equation*}
        K = \bigcap_{H \in \mathcal{A}} H
    \end{equation*}
    Clearly $1 \in K$.\\
    Given $a, b \in K$. Then $\forall H \in \mathcal{A}, a \in H, b \in H \rightarrow ab \in H$, i.e. $ab \in K$.\\
    Therefore $K$ is a subgroup of $G$.
\end{addmargin}
\subsection*{Definition of subgroup generated by $A$}
If $A$ is any subset of $G$, then define
\begin{equation*}
    \braket{A} = \bigcap_{A \subset H \leq G} H
\end{equation*}
This is called \textbf{subgroup of $G$ generated by $A$}.\\
From Proposition 8 we know that $\braket{A}$ is a subgroup, with $\mathcal{A} = \{H \leq G | A \subseteq H\}$. Since $\forall H, A \subseteq H \rightarrow A \subseteq \braket{A}$.\\
The uniqueness of $\braket{A}$ is as follows:
\begin{itemize}
    \item $\braket{A}$ is a subgroup that contains $A$.
    \item Any subgroup that contains $A$ also contains $\braket{A}$.
\end{itemize}
\subsection*{Properties of subgroup generated by $A$}
We try to define the same subgroup from bottom up:
\begin{equation*}
    \overline{A} = \{a_1^{\epsilon_1}a_2^{\epsilon_2}a_3^{\epsilon_3}...|n \in \mathbb{Z}, n \geq 0, a_ \in A, \epsilon_i = \pm 1\}
\end{equation*}
In this definition, $a_i$ need not be distinct.\\
From this definition we can see how a set is generated from $A$.
\subsection*{Proposition 9}
\begin{equation*}
    \overline{A} = \braket{A}
\end{equation*}
\textbf{Proof:}
\begin{addmargin}[1em]{0em}
    First obviously $\overline{A}$ is a subgroup.\\
    Since $\forall a = a_1^{\epsilon_1}a_2^{\epsilon_2}a_3^{\epsilon_3}...\in \overline{A}, \forall H \in \mathcal{A}, a \in H \rightarrow a \in \bigcap_{A \subseteq H \leq G} H \rightarrow a \in \braket{A}$.\\
    Therefore $\overline{A} \subseteq \braket{A}$.\\
    Since clearly $\overline{A}$ is also a subgroup that contains $A$, then $\braket{A} \subseteq \overline{A}$.\\
    Therefore $\overline{A} = \braket{A}$.
\end{addmargin}
\subsection*{Limitations of Subgroups Generated by $A$}
If $G$ is abelian, then we could collect the terms in $a_1^{\epsilon_1}a_2^{\epsilon_2}a_3^{\epsilon_3}...$ to re-define $\braket{A}$:
\begin{equation*}
    \braket{A} = \{a_1^{\alpha_1}a_2^{\alpha_2}...a_k^{\alpha_k}\}
\end{equation*}
If we further assume that each $a_i$ has finite order $d_i$, then we can bound the size of $\braket{A}$:
\begin{equation*}
    |\braket{A}| \leq d_1d_2...d_k
\end{equation*}
However, if $G$ is non-abelian, then the generated group can be much more complicated.\\
Let $G = D_8 = \braket{r, s}$ and choose $a = s, b = rs, A = \braket{a, b}$.\\
Since $r = rs\cdot s = ba, \rightarrow A = G, |A| = |G| = 8$. However, since $|a| = |b| = 2$, it is \textbf{impossible} to write all elements in $A$ as $a^\alpha b^\beta$.\\
The example above shows that for non-aeblian groups, the long product might not be collected. More specifically, the order of a finite group cannot be bounded even if we know the order of all the generating elements of the this group.\\
In some cases, the group generated by elements of finite orders could even have infinite order:\\
\begin{equation*}
    G = GL_2(\mathbb{R}), a = \begin{pmatrix}
        0 & 1\\
        1 & 0
    \end{pmatrix}, b = \begin{pmatrix}
        0 & 2\\
        1/2 & 0
    \end{pmatrix}
\end{equation*}
Now both $a, b$ have order $2$. But since $ab = \begin{pmatrix}
    1/2 & 0\\
    0 & 2
\end{pmatrix}$, it is clear that $\braket{a, b}$ is an infinite subgroup.
\section*{2.2 Centralizers and Normalizers, Stablizers and Kernels}
\subsection*{Definition of Centralizer}
Given set $A \subseteq G$, define
\begin{equation*}
    \mathbf{C}_G(A) = \{g \in G | \forall a \in A, gag^{-1} = a\}
\end{equation*}
This subset is called the \textbf{centralizer} of $A$ in $G$.\\
Since $gag^{-1} = a \iff ga = ag$, this is to say $\mathbf{C}_G(A)$ is the set of elements of $G$ which commutes with every elements of $A$.
\subsubsection*{Centralizers are Subgroups}
\textbf{Proof:}
\begin{addmargin}[1em]{0em}
    Obviously $1 \in \mathbf{C}_G(A)$.\\
    Suppose $x, y \in \mathbf{C}_G(A) \rightarrow \forall a \in A, xax^{-1} = a, yay^{-1} = a$.\\
    Therefore $xya(xy)^{-1} = xyay^{-1}x^{-1} = xax^{-1} = a$, i.e. $xy \in \mathbf{C}_G(A)$.\\
    This shows that $\mathbf{C}_G(A)$ is a subgroup of $G$.
\end{addmargin}
\subsection*{Definition of Center}
Define $Z(G) = \{g \in G | \forall x \in G, gx =xg\}$, i.e. the set of elements that commutes with all the elements of $G$. This is called the \textbf{center} of $G$.\\
Now it is clear that $Z(G) = \mathbf{C}_G(G)$ and therefore $Z(G) \leq G$, and is a special case of centralizer.
\subsection*{Definition of Normalizer}
Define $gAg^{-1} = \{gag^{-1}| a \in A\}$. Then the \textbf{normalizer} of $A$ in $G$ is
\begin{equation*}
    N_G(A) = \{g \in G | gAg^{-1} = A\}
\end{equation*}
Note that if $g \in \mathbf{C}_G(A)$ then $\forall g \in A, gag^{-1} = a$ and therefore $C_G(A) \leq N_G(A)$.\\
$N_G(A)$ can be proven to be a subgroup similar to $C_G(A)$.
\subsection*{Examples}
\begin{enumerate}
    \item If $G$ is abelian, then all elements of $G$ commutes and therefore $Z(G) = G$, which also collapses $C_G(A), N_G(A)$ to $G$.
    \item $G = D_8$ the dihedral group, with the usual definition of $r, s$. Then let $A = \{1, r, r^2, r^3\}$, WTS that $C_{D_8}(A) = A$:
    Given $sr^i \in D_8, sr^ir^j(sr^i)^{-1} = sr^{i+j}r^{-i}s = sr^js = r^{-j} \neq r^j$.\\
    Given $r^i \in D_8, r^ir^j(r^i)^{-1} = r^j$.\\
    Therefore $C_{D_8}(A) = A$
    \item In the previous example, we also WTS that $N_{D_8}(A) = D_8$:\\
    Since $C_G(A) = A \rightarrow A \leq N_G(A)$.\\
    $\forall r^i \in A, sr^is^{-1} = r^{-i}ss^{-1} = r^{-1} \rightarrow s \in N_G(A)$.\\
    Since $r, s \in N_G(A)$ and $N_G(A)$ is a group, this meaqns $N_G(A) = \braket{r, s} = G$.
    \item The center of $D_8$ is $\{1, r^2\}$:\\
    First we can see that for any $A \subseteq G, Z(G) \leq C_G(A) \leq N_G(A)$. Therefore $Z(G) \leq A = \{1, r, r^2, r^3\}$.\\
    Then given $r^i \in A, r^isr^j = sr^{-i}r^j$. In order for $r^isr^i = sr^{-i}r^j = sr^ir^j, \rightarrow r^i = r^{-i} \rightarrow r^i = r^2$.
    \item Let $G = S_3, A = \{1, (1\ 2)\}$. WTS $C_{S_3}(A) = N_{S_3}(A) = A$:\\
    Clearly $A \leq C_{S_3}(A)$. This, along with the fact that $C_{S_3}(A) \leq G$, and LaGrange's theorem, shows that $|C_{S_3}(A)|$ divides $2, 6$.\\
    i.e. $|C_{S_3}(A)| = 2$ or $6$. But since $(1\ 2\ 3)$ doesn't commute with $(1\ 2)$, this means $|C_{S_3}(A)| \neq 6 \rightarrow |C_{S_3}(A)| = 2 \rightarrow C_{S_3}(A) = A$.\\
    Then NTS $N_{S_3}(A) = A$, since $A$ contains only $(1\ 2)$ other than $1$. This means $\forall \sigma \in N_{S_3}(A), \sigma(1\ 2)\sigma^{-1} = (1\ 2)$ and therefore $\sigma \in C_{S_3}(A)$, i.e. $N_{S_3}(A) = C_{S_3}(A)$.
\end{enumerate}
\subsection*{Stabilizers and Kernels of Group Actions}
The normalizer, centralizer of $A$, and the center of $G$ are all just special cases of results on group actions.
\subsubsection*{Definition of Stabilizer}
Given $G$ a group action on a set $S$, then fixing an element $s \in S$, the \textbf{stabilizer} of $s$ in $G$ is the set
\begin{equation*}
    G_s = \{g \in G | g\cdot s = s\}
\end{equation*}
\subsubsection*{Stabilizer is a Subgroup}
The proof is obvious and similar to the proof that $C_G(A) \leq G$.
\subsection*{Definition of Kernel}
Given $G$ a group acting on a set $S$. Then the kernel of the action is defined as
\begin{equation*}
    \{g \in G| \forall s \in S, g\cdot s = s\}
\end{equation*}
similarly,
\subsubsection*{Kernels are Subgroups}
\subsection*{Relations between Centralizers, Normalizers and Stabilizers, Kernels*}
If we define $S = \mathcal{P}(G)$, the set of all subsets of $G$, then we can let $G$ act on $S$ by \textbf{conjugation}:
\begin{equation*}
    \forall g \in G, B \subseteq G, g\cdot B = gBg^{-1}
\end{equation*}
Then in this case, given $A \subseteq G, N_G(A) = $ the stabilizer of $A$.\\
Then if we let $N_G(A)$ act on $A$ also by conjugation, i.e.
\begin{equation*}
    \forall g \in N_G(A), a \in A, g\cdot a = gag^{-1}
\end{equation*}
Note this definition only makes sense when restricting the group to be $N_G(A)$ otherwise $gag^{-1}$ might not be in $A$.\\
Then in this case, $C_G(A)$ is just the kernel of this action.\\
Finally, if we let $G$ act on $S = G$ by conjugation, then $Z(G)$ is just the kernel of this action.
\end{document}