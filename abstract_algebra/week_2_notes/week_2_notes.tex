\documentclass{article}
\usepackage[utf8]{inputenc}
\usepackage{amsmath}
\usepackage{scrextend}
\usepackage{setspace}
\usepackage{amsfonts}
\usepackage{braket}

\title{Week 2 Notes\\
\large{Abstract Algebra}}
\author{shaozewxy }
\date{May 2022}

\doublespacing
\begin{document}

\maketitle

\setcounter{secnumdepth}{0}
\section{1.4 Matrix Groups}
\subsection{Defintion of a \textit{field}}
\begin{addmargin}[1em]{0em}
A \textit{field} is a set $F$ with two operations $+$ and $\times$,\\
such that $(F,+)$ is an abelian group, and $(F-\{0\}, \times)$ is also an abelian group,\\
while also satisfying the distribution rule:
\begin{equation*}
\forall a,b,c \in F, a\times(b+c) = a\times b+a\times c
\end{equation*}
\end{addmargin}
\subsection{Properties of Matrix Groups and Fields}
These facts will be proven later.
\begin{addmargin}[1em]{0em}
1. If $F$ is a field, and $F$ is finite, then $|F| = p^m$ for some $p, m \in \mathbb{Z}$ and $p$ a prime.\\
2. If $|F| = q$, then $|GL_n(F)| = (q^n-1)(q^n-q)(q^n-q^2)...(q^n-q^{n-1})$
\end{addmargin}

\section{1.5 The Quaternion Group}
\subsection{Definition of the \textit{Quaternion Group}}
\begin{addmargin}[1em]{0em}
The \textit{Quaternion Group} us defined as:
\begin{equation*}
    Q_8 = \{1, -1, i, j, k, -i, -j, -k\}
\end{equation*}
with product computed as:
\begin{equation*}
    \forall a \in Q_8, 1a = a1 = a
\end{equation*}
\begin{equation*}
    \forall a \in Q_8, (-1)(-1) = 1, (-1)a = a(-1) = -a
\end{equation*}
\begin{equation*}
    ii = jj = kk = -1
\end{equation*}
\begin{equation*}
    ij = k, ji = -k
\end{equation*}
\begin{equation*}
    jk = i, kj = -i
\end{equation*}
\begin{equation*}
    ki = j, ik = -j
\end{equation*}
\end{addmargin}
The Quaternion Group is the smallest non-abelian group.

\section{1.6 Homomorphisms and Isomorphisms}
\subsection{Examples}
\begin{addmargin}[1em]{0em}
1. For any group $G, G \cong G$.\\
While the identity map is an obvious isomorphism between $G$ and $G$, it is not necessarily the only such isomorphism.\\
2. The exponential map $\textrm{exp}: \mathbb{R} \rightarrow \mathbb{R}^+$ defined by $\textrm{exp}(x) = e^x$. This is an isomorphism from $(\mathbb{R}, +) \rightarrow (\mathbb{R}^+, \times)$.\\
This is so because this function has an inverse $\textrm{log}_e(x)$ and $e^xe^y = e^{x+y}$.\\
3. We show that the isomorphism type of a symmetric group is dependent only on the cardinality of the underlying set being permuted.\\
$\Delta, \Omega$ two nonempty sets with the same size, then $S_{\Delta} \cong S_{\Omega}$:\\
given $|\Delta| = |\Omega|$, then there exists a bijection $\theta$ between $\Delta, \Omega$. We can then define isomorphism $\phi: S_{\Delta} \rightarrow S_{\Omega}$ using $\theta$:
\begin{equation*}
    \phi(\sigma) = \sigma': \Omega \rightarrow \Omega, \sigma'(x) = \theta(\sigma(x))
\end{equation*}
Conversely, if $S_{\Delta} \cong S_{\Omega}$, then it is obvious that $|S_{\Delta}| = |S_{\Omega}|$, i.e., $|\Delta|! = |\Omega|!$, therefore $|\Delta| = |\Omega|$.\\
\end{addmargin}
\subsection{Properties about Isomorphisms}
\begin{addmargin}[1em]{0em}
1. Any non-abelian group of order $6$ is isomorphis to $S_3$.\\
In face, there are only two types of groups of order $6$, $S_3$ and $\mathbb{Z}/6\mathbb{Z}$.\\
2. If $\phi: G \rightarrow H$ is an isomorphism, then:\\
$|G| = |H|$\\
$G$ is abelian iff $H$ is ableian.\\
$\forall x \in G, |x| = |\phi(x)|$.\\
Using these rules, we can determine some groups are not isomorphic conveniently:\\
$S_3$ and $\mathbb{Z}/6\mathbb{Z}$ are not isomorphic because one is abelian and one is not.\\
$(\mathbb{R}-\{0\}, \times), (\mathbb{R}, +)$ are not isomorphic because $-1$ has order $2$ in $(\mathbb{R}-\{0\}, \times)$ while no elements in $(\mathbb{R}, +)$ has order $2$.\\
3. $G$ a finite group of order $n$ with $A=\{s_1, ..., s_m\}$ generating $G$. $H$ another group $B=\{r_1, ..., r_m\}$ elements of $H$. If any relation of $A$ is also satisfied by $B$ by replacing $s_i$ with $r_i$, then there is a unique homomorphism $\phi:G \rightarrow H$ which maps $s_i$ to $r_i$.\\
This means to check isomorphism, we only need to check the presentation of $G$:\\
If $B$ generates $H$, then $\phi$ is surjective, and if additionally, $|H| = |G|$, then $\phi$ is an isomorphism.\\
\subsubsection{Examples}
1. $D_{2n} = \braket{r,s|r^n=s^2=1, sr=r^{-1}s}$. Suppose $H$ a group containing elements $a,b$ with $a^n=b^2=1,ba=a^{-1}b$. Then $\exists$ homomorphism from $D_{2n}$ to $H$ mapping $r$ to $a$ and $s$ to $b$.\\
Given $k|n$, the $D_{2k}$ has a homomorphism to $D_{2n}$. Because $\{r_1, s_1\}$ generates $D_{2k}$, the homomorphism is surjective.\\
2. Between $D_6$ and $S_3$, with elements $a=(1 2 3), b=(1 2)$ satisfies $a^3 = 1, b^2 = 1, ba=ab^{-1}$. Then there is a homomorphism from $D_6$ to $S_3$ that sends $r \rightarrow a$, $s \rightarrow b$. Because $S_3$ is generated by $a, b$ and $|S_3| = |D_6|$, then $D_6 \cong S_3$.
\end{addmargin}
\end{document}