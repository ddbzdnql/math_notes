\documentclass{article}
\usepackage[utf8]{inputenc}
\usepackage{amsmath}
\usepackage{scrextend}
\usepackage{setspace}
\usepackage{amsfonts}
\usepackage{amssymb}
\usepackage{graphicx}
\graphicspath{ {../../images/} }

\title{Linear Algebra Done Right\\
\large{Week 4 Notes (b)}}
\author{shaozewxy }
\date{September 2022}

\doublespacing
\begin{document}

\maketitle

\setcounter{secnumdepth}{0}
\section*{5.C Eigenspaces and Diagonal Matrices}
\subsection*{Definition of eigenspace}
Given $T \in \mathcal{L}(V)$ and $\lambda \in \mathbf{F}$. The \textbf{eigenspace} of $T$ corresponding to $\lambda$ denoted $E(\lambda, T)$ is defined by
\begin{equation*}
    E(\lambda, T) = null(T - \lambda I)
\end{equation*}
This is to say, $E(\lambda, T)$ is the set of eigenvectors of $T$ corresponding to $\lambda$.
\subsection*{5.38 Sum of eigenspaces is a direct sum}
Suppose $V$ is finite-dimensional and $T \in \mathcal{L}(V)$ with $\lambda_1, ..., \lambda_m$ are distinct eigenvalues of $T$. Then
\begin{equation*}
    E(\lambda_1, T) + ... + E(\lambda_m, T)
\end{equation*}
is a direct sum. Furthermore,
\begin{equation*}
    dim\ E(\lambda_1, T) + ... + dim\ E(\lambda_m, T) \leq dim\ V
\end{equation*}
\textbf{Proof:}
\begin{addmargin}[1em]{0em}
    To prove $E(\lambda_i, T)$ are linearly independent, suppose $\exists v_1, ..., v_m$ such that $v_1 + ... + v_m = 0$, but since they are eigenvectors, it is impossible due to 5.10.\\
    Then $\sum dim\ E(\lambda_i, T) = dim(\bigoplus E(\lambda_i, T)) \leq dim\ V$
\end{addmargin}
\subsection*{Defintion of diagonalizable}
$T \in \mathcal{L}(V)$ is called \textbf{diagonalizable} if $T$ has a diagonal matrix w.r.t. some basis of $V$.
\subsection*{Example of diagonalizable operator}
Define $T \in \mathcal{T}(\mathbf{R}^2)$ by
\begin{equation*}
    T(x, y) = (41x + 7y, -20x + 74y)
\end{equation*}
$T$ with basis $(1, 4)(7, 5)$ has a matrix
\begin{equation*}
    \begin{pmatrix}
        69 & 0\\
        0 & 46
    \end{pmatrix}
\end{equation*}
Given $(x, y) \in \mathbb{R}^2$, we have
\begin{equation*}
    \begin{pmatrix}
        1 & 7 & x\\
        4 & 5 & y
    \end{pmatrix} \Rightarrow
    \begin{pmatrix}
        1 & 0 & \frac{7y-5x}{23}\\
        0 & 1 & \frac{4x-y}{23}
    \end{pmatrix}
\end{equation*}
Then
\begin{equation*}
    \begin{pmatrix}
        1 & 7\\
        4 & 5
    \end{pmatrix} \cdot
    \begin{pmatrix}
        69 & 0\\
        0 & 46
    \end{pmatrix} \cdot
    \begin{pmatrix}
        \frac{7y - 5x}{23}\\
        \frac{4x- y}{23}
    \end{pmatrix} =
    \begin{pmatrix}
        41x + 7y\\
        -20x + 74y
    \end{pmatrix}
\end{equation*}
\subsection*{5.41 Conditions for diagonalizability}
Given $V$ finite dimensional with $T \in \mathcal{L}(V)$. Let $\lambda_1, ..., \lambda_m$ denote the distinct eigenvalues of $T$. Then the following are equivalent:
\begin{itemize}
    \item[(a)] $T$ diagonalizable.
    \item[(b)] $V$ has a basis consisting of eigenvectors of $T$.
    \item[(c)] There exists 1-dimensional subspaces $U_1, ..., U_n$ of $V$ each invariant under $T$ such that
    \begin{equation*}
        V = U_1 \oplus ... \oplus U_n
    \end{equation*}
    \item[(d)] $V = \bigoplus E(\lambda_i, T)$
    \item[(e)] $dim\ V = \sum dim\ E(\lambda_i, T)$
\end{itemize}
\textbf{Proof:}
\begin{addmargin}[1em]{0em}
    First it is clear that $a \iff b$.\\
    Now $b \rightarrow c$ since each $U_i$ can just be $span(v_i)$.\\
    Similarly, suppose $c$ holds, picking $v_i \neq 0$ from $U_i$ creates a basis of $V$. Therefore $c \rightarrow b$.\\
    We have now proven that $a \iff b \iff c$. Next WTS that $b \rightarrow d \rightarrow e \rightarrow b$:\\
    Suppose $b$ holds: then $\forall v \in V, v = \sum a_iv_i$. Since $E(\lambda_i, T)$ are linearly independent, we know that $V = \bigoplus E(\lambda_i, T)$. THerefore $b \rightarrow d$.\\
    Suppose $d$ holds: then obviously $\sum dim\ E(\lambda_i, T) = dim\ \bigoplus E(\lambda_i, T) = dim\ V$. THerefore $d \rightarrow e$.\\
    Suppose $e$ holds: then for each $E(\lambda_i, T)$, pick $dim\ E(\lambda_i, T)$ independent vectors from it. This forms a basis of $V$ since the result is still independent and the size equals to $dim\ V$. Obviously this basis consist of eigenvectors. Therefore $e \rightarrow b$.
\end{addmargin}
Now every operator is diagonalizable even in complex space.
\subsection*{Example of un-diagonalizable operator}
The operator $T \in \mathcal{L}(\mathbf{C}^2)$ defined by
\begin{equation*}
    T(w, z) = (z, 0)
\end{equation*}
is not diagonalizable.\\
\textbf{Proof:}
\begin{addmargin}[1em]{0em}
    Suppose $Tv = T(w, z) = \lambda v = (\lambda w, \lambda z)$, then we have
    \begin{equation*}
        \begin{cases}
            \lambda z = 0\\
            \lambda w = z
        \end{cases}
    \end{equation*}
    If $\lambda \neq 0$, then $z = w = 0$.
\end{addmargin}
\subsection*{5.44 Enough eigenvalues implies diagonalizability}
If $T \in \mathcal{L}(V)$ has $dim\ V$ distinct eigenvalues then $T$ diagonalizable.\\
This is obvious since the corresponding eigenvectors form a basis of $V$.
\subsection*{5.45 Example of finding diagonal matrices}
Define $T \in \mathcal{L}(\mathbf{F}^3)$ by
\begin{equation*}
    T(x, y, z) = (2x+y, 5y+z, 8z)
\end{equation*}
Then it has a basis of eigenvectors.\\
\begin{addmargin}[1em]{0em}
    The matrix w.r.t. standard basis is upper triangular, therefore we have
    \begin{equation*}
        \begin{cases}
            2x = 2x+y\\
            2y = 5y+3z\\
            2z = 8z
        \end{cases} \Rightarrow \begin{pmatrix}
            1\\
            0\\
            0
        \end{pmatrix}
    \end{equation*}
    \begin{equation*}
        \begin{cases}
            5x = 2x+y\\
            5y = 5y+3z\\
            5z = 8z
        \end{cases} \Rightarrow \begin{pmatrix}
            3\\
            1\\
            0
        \end{pmatrix}
    \end{equation*}
    \begin{equation*}
        \begin{cases}
            8x = 2x+y\\
            8y = 5y+3z\\
            8z = 8z
        \end{cases} \Rightarrow \begin{pmatrix}
            1\\
            6\\
            6
        \end{pmatrix}
    \end{equation*}
    These are a basis of eigenvectors for $V$.
\end{addmargin}
\end{document}