\documentclass{article}
\usepackage[utf8]{inputenc}
\usepackage{amsmath}
\usepackage{scrextend}
\usepackage{setspace}
\usepackage{amsfonts}
\usepackage{amssymb}

\title{Linear Algebra Done Right\\
\large{Week 2 Notes}}
\author{shaozewxy }
\date{May 2022}

\doublespacing
\begin{document}

\maketitle

\setcounter{secnumdepth}{0}
\section{3.B Null Spaces and Ranges}
\subsubsection{\textit{Null Space}}
FOr $T \in L(V,W)$, the \textit{null space} of $T$, denoted $\textrm{null } T$ is the subset of $V$ the elements of which maps to $0$ through $T$:
\begin{equation*}
    \textrm{null } T = \{v\in V | Tv = 0\}
\end{equation*}
\subsection{Null space is a subspace}
\textbf{Proof:}
\begin{addmargin}[1em]{0em}
$T$ a linear map, therefore $T(0) = 0$, i.e., $0 \in null T$.\\
$\forall u, v \in null T, T(u+v) = Tu + Tv = 0+0 = 0$, i.e., $null T$ is closed under addition.\\
$\forall \lambda \in F, u \in null T, T(\lambda u) = \lambda Tu = \lambda 0 = 0$, i.e., $null T$ is closed under scalar multiplication.
\end{addmargin}
\subsection{Injectivity and Null Spaces}
$T \in L(V, W)$, then $T$ is injective iff $null T = \{0\}$.\\
\textbf{Proof:}\\
\begin{addmargin}[1em]{0em}
If $T$ injective, then $\forall v \in null T, Tv = 0 = T0 \rightarrow v = 0$.\\
If $null T = \{0\}$, then $\forall u, v \in T$ such that $Tu = Tv$, we have $T(u-v) = Tu - Tv = 0 \rightarrow u - v = 0$, i.e., $u = v$.
\end{addmargin}
\subsection{Fundamental Theorem of Linear Maps}
$V$ a finite dimensional space and $T \in L(V,W)$ then:
\begin{equation*}
    dim\ V = dim\ null\ T + dim\ range\ T
\end{equation*}
\textbf{Proof:}
\begin{addmargin}[1em]{0em}
$u_1, ..., u_m$ denotes a basis of $null\ T$, so $dim\ null\ T = m$. We can extend this basis to a basis of $V:\{u_1, .., u_m; v_1, ..., v_n\}$\\
Then we only need to show that $dim\ range\ T=n$. To do this, we prove that $B = \{Tv_1, ..., Tv_n\}$ is a basis of $range\ T$.\\
First we show that $B$ is independent:\\
If $\exists$ a linear combination of $B$ that sums up to $0$:
\begin{equation*}
    \sum_{i=1}^{n}a_i Tv_i = 0
\end{equation*}
Then we have:
\begin{equation*}
    T\left(\sum_{i=1}^{n} a_i v_i\right) = 0
\end{equation*}
i.e., $\sum_{i=1}^{n} a_i v_i \in null\ T$.\\
This means $\exists$ a linear combination of $u_1, ..., u_m$ that sums up to $\sum_{a_i}^{v_i}$:
\begin{equation*}
    \sum_{i=1}^{m} b_i u_i = \sum_{i=1}^{m} a_i v_i
\end{equation*}
\begin{equation*}
    \sum_{i=1}^{m} b_i u_i - \sum_{i=1}^{m} a_i v_i = 0
\end{equation*}
making the basis $u_1, .., u_m, v_1, ..., v_n$ dependent, contradiction.\\
Therefore, $Tv_1, ..., Tv_m$ must be independent.\\
We then show that $Tv_1, ..., Tv_m$ spans $range\ T$:\\
$\forall w \in range\ T, \exists v \in V$ such that $Tv = w$.\\
Because $u_1, ..., u_m, v_1, ..., v_n$ is a basis of $V$, then $\exists$ a linear combination of the basis that sums up to $v$:
\begin{equation*}
    \sum_{i=1}^{m} b_i u_i + \sum_{i=1}^{m} a_i v_i = v
\end{equation*}
Then we have:
\begin{equation*}
    T\left(\sum_{i=1}^{m} b_i u_i + \sum_{i=1}^{m} a_i v_i\right) = w
\end{equation*}
\begin{equation*}
    T\left(\sum_{i=1}^{m} b_i u_i\right) + T\left(\sum_{i=1}^{m} a_i v_i\right) = w
\end{equation*}
Because $u_1, ..., u_m$ a basis of $null\ T$, we have:
\begin{equation*}
    T\left(\sum_{i=1}^{m} b_i u_i\right) = 0
\end{equation*}
And therefore,
\begin{equation*}
    T\left(\sum_{i=1}^{m} a_i v_i\right) = w
\end{equation*}
\begin{equation*}
    \sum_{i=1}^{m} a_i Tv_i = w
\end{equation*}
i.e., $\forall w \in range\ T, \exists$ a linear combination of $B$ that sums to $w$, making $B$ spanning $range\ T$.\\
Therefore, $B$ is a basis.
\end{addmargin}
\subsubsection{Dimensions and In/Surjectivity}
$V, W$ both finite-dimensional, with $dim\ V > dim\ W$, then $\nexists T\in L(V, W)$ that is injective.\\
\textbf{Proof:}
\begin{addmargin}[1em]{0em}
$\forall T \in L(V,W), dim\ range\ T + dim\ null\ T = dim\ V$, also because $dim\ range\ T \leq dim\ W$, we have
\begin{equation*}
    dim\ V - dim\ null\ T \leq dim\ W
\end{equation*}
\begin{equation*}
    dim\ null\ T \geq dim\ V - dim\ W \geq 0
\end{equation*}
Therefore, $T$ can't be injective.\\
Similarly, $V, W$ both finite-dimensional, with $dim\ V < dim\ W$, then $\nexists T\in L(V, W)$ that is surjective.\\
The proof is similar.
\end{addmargin}
\subsubsection{Dimensions and Linear Systems}
Translating the two lemmas above to linear systems, we get:
\begin{addmargin}[1em]{0em}
A homogeneous system of linear equations with more variables than equations has nonzero solutions.\\
An inhomogeneous system of linear equations with more equations than variables has no solutions for some choices of the constant terms.
\end{addmargin}

\section{3.D Invertibility and Isomorphic Vector Spaces}
\subsection{Invertibility, Injectivity and Surjectivity}
\begin{addmargin}[1em]{0em}
A linear map is invertible iff it is both injective and surjective.
\end{addmargin}
\textbf{Proof:}
\begin{addmargin}[1em]{0em}
$T \in L(V, W)$ is invertible,\\
Show $T$ is injective:\\
$\forall a, b \in V$ such that $Ta = Tb$, then $T^{-1}(Ta) = =a = T^{-1}(Tb) = b$, therefore $T$ is injective.\\
Show $T$ is surjective:\\
$\forall w \in W$, let $v = T^{-1}w$, then $Tv = T(T^{-1}w) = w$, therefore $T$ is surjective.\\
$T \in L(V, W)$ is both injective and surjective, WTS $T$ has an inverse.\\
Define $T^{-1}:W \rightarrow V$, $\forall w \in W$, because $T$ surjective, $\exists v \in V$ such that $Tv = w$, let $T^{-1}w$ to be such $v$.\\
$\forall v \in V. T(T^{-1}Tv) = Tv$, because $T$ injective, $\exists!$ therefore $T^{-1}Tv = v$.
\end{addmargin}
\subsection{Isomorphism}
\subsubsection{Dimensions and Isomorphism}
\begin{addmargin}[1em]{0em}
Two finite-dimensional vector spaces over $F$ are isomorphic iff they have the same dimensions.
\end{addmargin}
\textbf{Proof:}
\begin{addmargin}[1em]{0em}
If $V \cong W$ and both finite-dimensional, therefore, $\exists T:V \rightarrow W$ such that $T$ is invertible.\\
i.e. $T$ is injective and surjective.\\
We can infer from this that $dim\ V = dim\ W$.\\
If $dim\ V = dim\ W = n$, then $A = v_1, ..., v_n$ and $B = w_1, ..., w_n$ bases for $V, W$.\\
Define $T:V \rightarrow W$ as $Tv_i = w_i$.\\
WTS $T$ is injective:\\
$T(\sum_{i=1}^{n}a_iv_i) = T(\sum_{i=1}^{n}b_iv_i)$, then $\sum_{i=1}^{n}a_iw_i = \sum_{i=1}^{n}b_iw_i$, but because $w_1, ..., w_i$ is a basis, then $a_i = b_i$ must be true, i.e. $\sum_{i=1}^{n}a_iv_i = \sum_{i=1}^{n}b_iv_i$. Therefore, $T$ is injective.\\
WTS $T$ is surjective:\\
$\forall \sum_{i=1}^{n}a_iw_i \in W$, let $v = \sum_{i=1}^{n}a_iv_i \in V$, then $Tv = T(\sum_{i=1}^{n}a_iv_i) = \sum_{i=1}^{n}a_iTv_i = \sum_{i=1}^{n}a_iw_i = w$. Therefore, $T$ is surjective.\\
Therefore $T$ is invertible.
\end{addmargin}
\subsubsection{$L(V,W)$ and $\textbf{F}^{m,n}$ are isomorphic}
\begin{addmargin}[1em]{0em}
$A = v_1, ..., v_n$ a basis of $V$ and $B = w_1, .., w_m$ a basis of $W$. The $M(V, W, A, B)$ defines an isomorphism between $L(V, W)$ and $\textbf{F}^{m,n}$.
\end{addmargin}
\textbf{Prood:}
\begin{addmargin}[1em]{0em}
WTS $M$ is injective:\\
$T\in L(V,W)$ and $M(T) = 0$, then we know $Tv_k = 0$ and because $v_1, ..., v_n$ is a basis, $T = 0$, therefore $M$ is injective.\\
WTS $M$ is surjective:\\
This is obvious, given $A \in \textbf{F}^{m,n}$, define $T$ accordingly.
\end{addmargin}
\subsection{Linear Map and Matrix Multiplication}
\subsubsection{Linear maps act like matrix multiplication}
\begin{addmargin}[1em]{0em}
$T\in L(V,W)$ and $v\in V$, with $A = v_1, ..., v_n$ and $B = w_1, ..., w_n$ bases of $V$ and $W$. Then
\begin{equation*}
    M(Tv) = M(T)M(v)
\end{equation*}
\end{addmargin}
\textbf{Proof:}
\begin{addmargin}[1em]{0em}
$\forall v \in V = \sum_{i=1}^{n}a_iv_i, M(Tv) = M(\sum_{i=1}^{n}a_iTv_i)$, then we have
\begin{equation*}
M(v) = \begin{bmatrix}
a_1\\
a_2\\
...\\
a_n
\end{bmatrix}, M(T) = \begin{bmatrix}
Tv_1 & Tv_2 & ... & Tv_n
\end{bmatrix}
\end{equation*}
\begin{equation*}
    M(Tv) = \sum_{i=1}^{n} a_iM(Tv_i) = \sum_{i=1}^{n} a_iM(T)_{\_i} = M(T)M(v)
\end{equation*}
\end{addmargin}
\subsection{Operators}
\begin{addmargin}[1em]{0em}
A linear map from a vector space to itself is called an \textit{operator}.
\end{addmargin}
\subsection{Properties in finite dimensions}
\begin{addmargin}[1em]{0em}
$V$ a finite-dimensional vector space and $T\in L(V)$, then the following are equivalent:\\
$T$ invertible.\\
$T$ injective.\\
$T$ surjective.
\end{addmargin}
The proof comes from the dimension formula:\\
$dim\ null\ T + dim\ range\ T = dim\ V = dim\ V$
\end{document}