\documentclass{article}
\usepackage[utf8]{inputenc}
\usepackage{amsmath}
\usepackage{scrextend}
\usepackage{setspace}
\usepackage{amsfonts}
\usepackage{amssymb}
\usepackage{graphicx}
\graphicspath{ {../../images/} }

\title{Linear Algebra Done Right\\
\large{HW 3}}
\author{shaozewxy }
\date{June 2022}

\doublespacing
\begin{document}

\maketitle

\setcounter{secnumdepth}{0}
\section*{3.10}
\includegraphics*[scale=0.5]{q3_2.png}
\subsection*{a.}
\begin{equation*}
    \begin{split}
        dim(range\ ST) = dim\ V - dim\ null\ ST\\
        dim(range\ T) = dim\ V - dim\ null\ T
    \end{split}
\end{equation*}
We just NTS that $dim\ null\ ST > dim\ null\ T$.\\
To do so, WTS that $null\ T \subseteq null\ ST$:\\
This is obvious since $\forall v \in null\ T, ST(v) = S(Tv) = S(0) = 0 \rightarrow v \in null\ ST$.\\
Therefore we have shown that $dim(range\ ST) \leq dim(range\ T)$.
\subsection*{b.}
From a. we know that the equality is achieved $\iff dim\ null\ T = dim\ null\ ST$, i.e. $null\ T = null\ ST$.\\
Now suppose $dim(range\ ST) = dim(range\ T)$, WTS $range\ T + null\ S = range\ T \oplus null\ S$.\\
Since we know $null\ T = null\ ST$, and obviously $range\ T$ independent with $null\ T$, we conclude that $range\ T$ independent with $null\ ST$ and therefore $range\ T + null\ S = range\ T \oplus null\ S$.\\
Then suppose $range\ T + null\ S = range\ T \oplus null\ S$, WTS $dim(range\ ST) = dim(range\ T)$:\\
Suppose $\exists v \in null\ ST - null\ T$, then this means $Tv = w \neq 0 \in null\ S$, i.e. $w \neq 0 \in null\ S \cap range\ T$, contradiction with the fact that $range\ T$ independent with $null\ S$. Therefore no such $v$ exists and thus $null\ ST = null\ T \Rightarrow dim(range\ ST) = dim(range\ T)$.
\subsection*{c.}
Given $v \in V$, if $v \in null\ T$, then $Tv = 0 \Rightarrow ST(v) = 0 \Rightarrow v \in null\ ST$. Therefore we know that $null\ S \subseteq null\ ST$.\\
Then we create $F \in \mathcal{L}(T^{-1}(null\ S), null\ S)$ defined as $Fv = Tv$.\\
By definition $T^{-1}(null\ S) = null\ ST$.\\
Obviously $null\ F = null\ T$.\\
Therefore we can conclude that
\begin{equation*}
    \begin{split}
        dim\ range\ F + dim\ null\ F &= dim(T^{-1}(null\ S))\\
        dim\ range\ F + dim\ null\ T &= dim\ null\ ST\\
        dim\ null\ ST &\leq dim\ null\ S + dim\ null\ T
    \end{split}
\end{equation*}
\subsection*{d.}
From c. we see that the equality is achived when $dim\ range\ F = dim\ null\ S$, i.e.
\begin{equation*}
    null\ S \subseteq range\ T
\end{equation*}
\end{document}