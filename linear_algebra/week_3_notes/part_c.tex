\documentclass{article}
\usepackage[utf8]{inputenc}
\usepackage{amsmath}
\usepackage{scrextend}
\usepackage{setspace}
\usepackage{amsfonts}
\usepackage{amssymb}
\usepackage{graphicx}
\graphicspath{ {../../images/} }

\title{Linear Algebra Done Right\\
\large{Week 3 Notes (c)}}
\author{shaozewxy }
\date{June 2022}

\doublespacing
\begin{document}

\maketitle

\setcounter{secnumdepth}{0}
\section*{3.E Products and Quotients of Vector Space}
\subsection*{Products of Vector Spaces}
\subsubsection*{3.71 Definition of product of vector spaces}
Given $V_1, ..., V_m$ are vector spaces over $\mathbf{F}$.
\begin{itemize}
    \item The \textbf{product} $V_1 \times ... \times V_m$ is defined by
    \begin{equation*}
        V_1 \times ... \times V_m = \{(v_1, ..., v_m): v_1 \in V_1, ..., v_m \in V_m\}
    \end{equation*}
    \item Addition of $V_1, ..., V_m$ is defined by
    \begin{equation*}
        (u_1, ..., u_m) + (v_1, ..., v_m) = (u_1 + v_1, ..., u_m + v_m)
    \end{equation*}
    \item Scalar multiplication is defined by
    \begin{equation*}
        \lambda(v_1, ..., v_m) = (\lambda v_1, ..., \lambda v_m)
    \end{equation*}
\end{itemize}
\subsubsection*{3.73 Product of vector spaces is a vector space}
The proof of this is obvious.
\subsubsection*{3.76 Dimension of a product is the sum of dimensions}
Given $V_1, ..., V_m$ finite-dimension vector spaces, then $V_1 \times ... \times V_m$ is also finite-dimensional add
\begin{equation*}
    dim(V_1 \times ... \times V_m) = dim\ V_1 + ... + dim\ V_m
\end{equation*}
\begin{addmargin}[1em]{0em}
    To prove this, just create a basis where all entris all $0$ except for a basis vector from on of the vector space.\\
    This is obviously a basis and the dimension is just the sum of dimensions of all the vector spaces.
\end{addmargin}
\subsection*{Products and Direct Sums}
\subsubsection*{3.77 Product and direct sums}
Given $U_1, ..., U_m$ subspaces of $V$. Define map $\Gamma: U_1 \times ... \times U_m \rightarrow U_1 + ... + U_m$ by
\begin{equation*}
    \Gamma(u_1, ..., u_m) = u_1 + ... + u_m
\end{equation*}
Then $U_1 + ... + U_m$ is a direct sum $\iff \Gamma$ is injective.\\
\textbf{Proof:}
\begin{addmargin}[1em]{0em}
    This is basically saying that $\nexists (u_1, ..., u_m) \in U_1 \times ... \times U_m$ such that $u_1 + ... + u_m = 0$, therefore making $U_1 + ... + U_m$ a direct sum.
\end{addmargin}
Since $\Gamma$ is naturally surjective, therefore we can say it is a direct sum $\iff \Gamma$ is invertible. Therefore, the below result
\subsubsection*{3.78 Condition for direct sum}
Given $V$ finite-dimensional and $U_1, ..., U_m$ subspaces of $V$. Then $U_1 + ... + U_m$ is a direct sum $\iff$
\begin{equation*}
    dim(U_1 + ... + U_m) = dim\ U_1 + ... + dim\ U_m
\end{equation*}
\subsection*{Quotient of Vector Spaces}
\subsubsection*{3.79 Definition of v+U}
Given $v \in V, U \leq V$. Then $v+U$ is a subset of $V$ defined by
\begin{equation*}
    v+U = \{v+u : u\in U\}
\end{equation*}
\subsubsection*{3.81 Definition of affine subset and parallel}
\begin{itemize}
    \item An \textbf{affine subset} of $V$ of the form $v+U$ for some $v \in V, U \leq V$.
    \item For $v \in V, U \leq V$, the affine subset $v+U$ is said to be \textbf{parallel} to $U$.
\end{itemize}
\subsubsection*{3.83 Definition of quotient space, V/U}
Given $U \leq V$. THen the quotient space $V/U$ is the set of all affine subsets of $V$ parallel $U$, i.e.
\begin{equation*}
    V/U = \{v+U : v \in V\}
\end{equation*}
Next we try to show that $V/U$ is a vector space.
\subsubsection*{3.85 Two affine subsets are equal or disjoint}
Given $U \leq V, v, w \in V$, the following are equivalent:
\begin{itemize}
    \item[(a)] $v-w \in U$
    \item[(b)] $v+U = w+U$
    \item[(c)] $(v+U) \cap (w+U) \neq \emptyset$
\end{itemize}
\textbf{Proof:}
\begin{addmargin}[1em]{0em}
    Suppose a is true, then we WTS b is also true:\\
    $\forall u'$ such that $\exists u \in U, v+u = u'$, we have that 
    \begin{equation*}
        v + u = w + v - w + u = w + (v-w + u) \in w+U
    \end{equation*}
    Therefore $v+U \subseteq w+U$, similarly $w+U \subseteq v+U$.\\
    Thereofre $v+U = w+U$.\\
    Now obviously that $b \rightarrow c$.\\
    We only NTS $c \rightarrow a$:\\
    Suppose $\exists u' \in v+U \cap w+U$ such that $\exists u_1, u_2 \in U, v+u_1 = u' = w+u_2$, then we have $v-w = u_2 - u_1 \in U$. Proving a.
\end{addmargin}
With the result above we can define the operations on $V/U$.
\subsubsection*{3.86 Definition of addition and scalar multiplication on V/U}
Given $U \leq V$. Then \textbf{addition} and \textbf{scalar multiplication} are defined on $V/U$ by
\begin{equation*}
    \begin{split}
        (v+U) + (w+U) = (v+w)+U\\
        \lambda (v+U) = (\lambda v+U)
    \end{split}
\end{equation*}
\subsubsection*{3.87 Quotient space is a vector space}
\textbf{Proof:}
\begin{addmargin}[1em]{0em}
    We need to show that the addition and multiplication above are well-defined.\\
    For addition, suppose $v+U = v'+U, w+U = w'+U$. we NTS $(v+U)+(w+U)=(v+w)+U = (v'+U)+(w'+U) = (v'+w')+U$:\\
    Since $v+U = v'+U, \rightarrow v' - v \in U$, similarly $w' - w \in U$.\\
    Therefore we have
    \begin{equation*}
        v' - v + w' - w = (v'+w') - (v+w) \in U
    \end{equation*}
    Therefore $(v'+w')+U = (v+w)+U$.\\
    The scalar multiplication can be seen as $\lambda$ times of $+ (v+U)$, therefore is already proven.
\end{addmargin}
\subsubsection*{Definition of quotient map, $\pi$}
Given $U \leq V$, the \textbf{quotient map} $\pi:V \rightarrow V/U$ is defined by
\begin{equation*}
    \forall v \in V, \pi(v) = v+U
\end{equation*}
\subsubsection*{3.89 Dimension of quotient space}
Given $U \leq V$, then
\begin{equation*}
    dim\ V/U = dim\ V - dim\ U
\end{equation*}
\textbf{Proof:}
\begin{addmargin}[1em]{0em}
    \begin{equation*}
        \begin{split}
            dim\ V/U &= dim\ range\ \pi\\
            &= dim\ V - dim\ null\ \pi\\
            &= dim\ V - dim\ U
        \end{split}
    \end{equation*}
\end{addmargin}
\subsubsection*{3.90 Definition of $\tilde{T}$}
Given $T \in \mathcal{L}(V, W)$, define $\tilde{T}: V/(null\ T) \rightarrow W$ by
\begin{equation*}
    \tilde{T}(v + null\ T) = Tv
\end{equation*}
NTS that the definition makes sense:
\begin{addmargin}[1em]{0em}
    Given $v_1, v_2 \in V$ such that $v_1 + null\ T = v_2 + null\ T$, then:\\
    \begin{equation*}
        \tilde{T}v_1 - \tilde{T}v_2 = Tv_1 -Tv_2 = T(v_1 - v_2)
    \end{equation*}
    Now since $v_1 + null\ T = v_2 + null\ T, \rightarrow v_1 - v_2 \in null\ T$.\\
    Therefoore $\tilde{T}v_1 - \tilde{T}v_2 = T(v_1 - v_2) = 0$.
\end{addmargin}
\subsubsection*{3.91 Null space and range of $\tilde{T}$}
Given $T \in \mathcal{L}(V, W)$, then we have:
\begin{itemize}
    \item[(a)] $\tilde{T}$ is injective
    \item[(b)] $range\ \tilde{T} = range\ T$
    \item[(c)] $V/(null\ T)$ is isomorphic to $range\ T$
\end{itemize}
\textbf{Proof:}
\begin{addmargin}[1em]{0em}
    \begin{itemize}
        \item[(a)] Given $\tilde{a}, \tilde{b} \in V/U$ such that $\tilde{T}(\tilde{a}) = \tilde{T}(\tilde{b})$, then we have $Ta = Tb \rightarrow a-b \in null\ T \rightarrow \tilde{a} = \tilde{b}$.
        \item[(b)] This is obvious.
        \item[(c)] This comes from a and b.
    \end{itemize}
\end{addmargin}
\end{document}